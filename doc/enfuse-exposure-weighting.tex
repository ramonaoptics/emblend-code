%% This file is part of Enblend.
%% Licence details can be found in the file COPYING.


\section[Exposure Weighting]{\label{sec:exposure-weighting}%
  \genidx{weighting!exposure}%
  \gensee{exposure weighting}{weighting, exposure}%
  Exposure Weighting}

\optidx{--exposure-optimum}%
Exposure weighting prefers pixels with a luminance~$Y$ close to the user-chosen optimum value
(option~\option{--exposure-optimum}, default: \val{val:default-exposure-optimum}) of the
normalized, real-valued luminance interval~$(0, 1)$.

\genidx{projector!grayscale}%
\gensee{grayscale projector}{projector, grayscale}%
\optidx{--gray-projector}%
\acronym{RGB}-pixels get converted to luminance before using the grayscale projector given by
\sample{--gray-projector}, which defaults to \code{average}.  Grayscale pixels simply are
identified with luminance.

\genidx{luminance interval!normalized}%
\gensee{normalized luminance interval}{luminance interval, normalized}%
In the normalized luminance interval 0.0 represents pure black and 1.0 represents pure white
independently of the data type of the input image.  This is, for a \acronym{JPEG} image the
luminance~255 maps to 1.0 in the normalized interval and for a 32~bit \acronym{TIFF} picture the
highest luminance value~4294967295 also maps to 1.0.  The middle of the luminance interval, 0.5,
is where a neutral gray tone ends up with every camera that had no exposure correction dialed
in, for example the image of any gray-card or white-card.

The exposure weighting algorithm only looks at a single pixel at a time; the pixel's
neighborhood is not taken into account.


\subsection[Built-In Functions]{\label{sec:built-in-functions}%
  \genidx{weighting!exposure!built-in}%
  Built-In Functions}

\genidx{exposure weight function!\code{gauss}}%
\genidx{exposure weight function!\propername{Gaussian}}%
Up to \App{} version~4.1 the only weighting function is the \propername{Gaussian}
\begin{equation*}\refrep{equ:weight:gauss}%
    w_{\mathrm{exp}}(Y) =
    \exp\left(-\frac{1}{2}
              \left( \frac{Y - Y_{\mathrm{opt}}}{\mathit{width}} \right)^2\right),
\end{equation*}
\noindent whose maximum position~$Y_{\mathrm{opt}}$ and $width$ are controlled by the command
line options \option{--exposure-optimum} and~\option{--exposure-width} respectively, where
$Y_{\mathrm{opt}}$ defaults to \val{val:default-exposure-optimum} and $width$ defaults to
\val{val:default-exposure-width}.  \figureName~\ref{fig:gaussian} shows some
\propername{Gaussians}.


\begin{figure}
  \ifreferencemanual\begin{maxipage}\fi
  \centering
  \includeimage{gaussian}
  \ifreferencemanual\end{maxipage}\fi

  \caption[\propername{Gaussian} weight function]{\label{fig:gaussian}%
    \App{}'s \propername{Gaussian} function with the parameters \metavar{optimum} = 0.5 and
    three different \metavar{width}s: 0.1, 0.2, and~0.4.}
\end{figure}


\optidx{--exposure-optimum}%
\optidx{--exposure-width}%
The options \option{--exposure-optimum} and~\option{--exposure-width} serve to fine-tune the
final result without changing the set of input images.  Option~\option{--ex\shyp po\shyp
  sure\hyp op\shyp ti\shyp mum} sets the point of optimum exposure.  Increasing the
\metavar{optimum} makes \App{} prefer lighter pixels, rendering the final image lighter, and
vice versa.  Option~\option{--exposure-width} defines the \metavar{width} of acceptable
exposures.  Small values of \metavar{width} penalize exposures that deviate from
\metavar{optimum} more, and vice versa.

\optidx{--exposure-weight-function}%
In \App{} version~4.2 several new exposure weight functions have been added.  Select them with
option~\option{--exposure-weight-function}.  For the following presentation we refer to the
linear luminance transform
\begin{equation*}\refrep{equ:linear-luminance-transform}
  z = \frac{Y - Y_{\mathrm{opt}}}{\mathit{width}}.
\end{equation*}
as introduced in \equationabbr~\fullref{equ:linear-luminance-transform}.

\begin{table}
  \ifreferencemanual\begin{maxipage}\fi
  \centering
  \begin{tabular}{p{.28\textwidth}lcc}
    \hline
    \multicolumn{1}{c|}{Exposure Weight} &
    \multicolumn{1}{c|}{\metavar{WEIGHT-FUNC.}} &
    \multicolumn{1}{c|}{Equ.} &
    \multicolumn{1}{c}{Chart} \\
    \hline\extraheadingsep
    \propername{Gaussian} curve (default)%
    \genidx{exposure weight function!\code{gauss}}%
    \genidx{exposure weight function!\propername{Gaussian}}%
    & \code{gauss}, \code{gaussian} &
    \fullref*{equ:weight:gauss} &
    \fullref*{fig:gaussian} \\
    \propername{Lorentz} curve%
    \genidx{exposure weight function!\code{lorentz}}%
    \genidx{exposure weight function!\propername{Lorentz} curve}%
    \genidx{exposure weight function!\propername{Lorentzian}}%
    & \code{lorentz}, \code{lorentzian} &
    \fullref*{equ:weight:lorentz} &
    \fullref*{fig:lorentzian} \\
    Upper half-wave of a sine%
    \genidx{exposure weight function!\code{halfsine}}%
    \genidx{exposure weight function!\code{half-sine}}%
    & \code{halfsine}, \code{half-sine} &
    \fullref*{equ:weight:halfsine} &
    \fullref*{fig:halfsine} \\
    Full sine-wave shifted upwards by one%
    \genidx{exposure weight function!\code{fullsine}}%
    \genidx{exposure weight function!\code{full-sine}}%
    & \code{fullsine}, \code{full-sine} &
    \fullref*{equ:weight:fullsine} &
    \fullref*{fig:fullsine} \\
    Quartic, or bi-square function%
    \genidx{exposure weight function!\code{bisquare}}%
    \genidx{exposure weight function!\code{bi-square}}%
    & \code{bisquare}, \code{bi-square} &
    \fullref*{equ:weight:bisquare} &
    \fullref*{fig:power}
  \end{tabular}
  \ifreferencemanual\end{maxipage}\fi

  \caption[Exposure weight functions]{\label{tab:weight-functions}%
    \genidx{exposure weight functions}%
    Available, compiled-in exposure weight functions.}
\end{table}


Functions \propername{Gaussian}
\begin{equation*}\refrep{equ:weight:gauss}%
    w_{\mathrm{exp}}(z) = \exp\left(-z^2 / 2\right)
\end{equation*}
and \propername{Lorentzian}
\begin{equation*}\refrep{equ:weight:lorentz}
  w_{\mathrm{exp}}(z) = \frac{1}{1 + z^2 / 2}
\end{equation*}
\noindent behave like $1 - z^2$ around the optimum.  However for large $|z|$ the
\propername{Gaussian} weight rolls off like $\exp(-z^2/2)$ and the \propername{Lorentzian} only
as $z^{-2}$.


\begin{figure}
  \ifreferencemanual\begin{maxipage}\fi
  \centering
  \includeimage{lorentzian}
  \ifreferencemanual\end{maxipage}\fi

  \caption[\propername{Lorentzian} function]{\label{fig:lorentzian}%
    \App{}'s \propername{Lorentzian} function with the parameters $\metavar{optimum} = 0.5$ and
    three different \metavar{width}s: 0.1, 0.2, and~0.4.}
\end{figure}


Both, the \propername{Gaussian} and the \propername{Lorentzian} are easy to use, because they do
not go exactly to zero.  Thus, \App{} can select ``better'' pixels even far away from the chosen
optimum.


\begin{figure}
  \ifreferencemanual\begin{maxipage}\fi
  \centering
  \includeimage{halfsine}
  \ifreferencemanual\end{maxipage}\fi

  \caption[Half-Sine function]{\label{fig:halfsine}%
    \App{}'s Half-Sine function with the parameters $\metavar{optimum} = 0.5$ and three
    different \metavar{width}s: 0.1, 0.2, and~0.4.}
\end{figure}


\begin{figure}
  \ifreferencemanual\begin{maxipage}\fi
  \centering
  \includeimage{fullsine}
  \ifreferencemanual\end{maxipage}\fi

  \caption[Full-Sine function]{\label{fig:fullsine}%
    \App{}'s Full-Sine function with the parameters $\metavar{optimum} = 0.5$ and three
    different \metavar{width}s: 0.1, 0.2, and~0.4.}
\end{figure}


Again, Half-Sine
\begin{equation}\refrep{equ:weight:halfsine}
  w_{\mathrm{exp}}(z) =
  \left\{\begin{array}{cl}
  \cos(z) & \mbox{if } |z| \leq \pi/2 \\
  0       & \mbox{otherwise.}
  \end{array}\right.
\end{equation}
and Full-Sine
\begin{equation}\refrep{equ:weight:fullsine}
  w_{\mathrm{exp}}(z) =
  \left\{\begin{array}{cl}
  (1 + \cos(z)) / 2 & \mbox{if } |z| \leq \pi \\
  0                 & \mbox{otherwise.}
  \end{array} \right.
\end{equation}
\noindent behave like $1 - z^2$ around the optimum, like \propername{Gaussian} and
\propername{Lorentzian}.  However for large $|z|$ they both are exactly zero.  The difference is
how they decrease just before they reach zero.  Half-Sine behaves like $z - z'$ and Full-Sine
like $(z - z'')^2$, where $z'$ and $z''$ are the respective zeros.


\begin{figure}
  \ifreferencemanual\begin{maxipage}\fi
  \centering
  \includeimage{power}
  \ifreferencemanual\end{maxipage}\fi

  \caption[Bi-Square function]{\label{fig:power}%
    \App{}'s Bi-Square function with the parameters $\metavar{optimum} = 0.5$ and three
    different \metavar{width}s: 0.1, 0.2, and~0.4.}
\end{figure}


Bi-Square
\begin{equation}\refrep{equ:weight:bisquare}
  w_{\mathrm{exp}}(z) =
  \left\{
  \begin{array}{cl}
    1 - z^4 & \mbox{if } |z| \leq 1 \\
    0       & \mbox{otherwise.}
  \end{array}
  \right.
\end{equation}
\noindent is the only predefined function that behaves like $1 - z^4$ around the optimum.

The weight functions Half-Sine, Full-Sine, and Bi-Square are more difficult to use, because they
yield exactly zero if the normalized luminance of a pixel is far enough away from the optimum.
This can lead to pathologies if the luminances of the same pixel position in all $N$ input
images are assigned a weight of zero.  For all-zero weights \App{} falls back to weighing
equally.  This is, each pixel gets a weight of $\slfrac{1}{N}$, which may or may not be the
desired result.  However, if the \metavar{width} is chosen so large that the weight never
vanishes or the input images span a large enough range of luminances for each and every pixel,
the problem is circumnavigated.

\optidx{--exposure-cutoff}%
Another way of cutting off underexposed or overexposed pixels is to use
option~\option{--exposure-cutoff}, which has the additional benefit of allowing to choose upper
and lower cutoff separately.


\begin{figure}
  \ifreferencemanual\begin{maxipage}\fi
  \centering
  \includeimage{exposure-weights}
  \ifreferencemanual\end{maxipage}\fi

  \caption[Comparison of exposure weight functions]{\label{fig:exposure-weights}%
    Comparison of all of \App{}'s built-in exposure weight functions~$w(Y)$ for the default
    values of $\metavar{optimum} = 0.5$ and $\metavar{width} = 0.2$.  Note that all functions
    intersect at $w = \slfrac{1}{2}$, this is, they share the same \acronym{FWHM}.}
\end{figure}


\figureName~\ref{fig:exposure-weights} compares all available exposure weight functions for the
same parameters, namely their defaults.  They all intersect at $w = \slfrac{1}{2}$ independently
of \metavar{optimum} or \metavar{width}, making it simple to just try them out without
fundamentally changing brightness.


\subsection[User-Defined Functions]{\label{sec:user-defined-functions}%
  \genidx{weighting!exposure!user-defined}%
  User-Defined Exposure Weighting Functions}

Depending on how \App{} was compiled it may support dynamically-linked exposure weighting
functions, \acronym{OpenCL} exposure weighting functions, or both.

%% This file is part of Enblend.
%% Licence details can be found in the file COPYING.


\subsection[User-Defined Dynamic Functions]{\label{sec:user-defined-dynlink-functions}%
  \genidx{weighting!exposure!user-defined!dynamically-linked}%
  User-Defined, Dynamically-Linked Exposure Weighting Functions}

See also \sectionName~\ref{sec:user-defined-opencl-functions} below on defining weighting
functions via \acronym{OpenCL}.

\restrictednote{\acronym{Dynamic Linking}-enabled versions only.}

\genidx{linking!dynamic}%
\gensee{dynamic linking}{linking, dynamic}%
\genidx{loading!dynamic}%
\gensee{dynamic loading}{loading, dynamic}%
\genidx{shared object}%
\gensee{object!shared}{shared object}%
\genidx{dynamic library}%
\gensee{library!dynamic}{dynamic library}%
On operating systems, where dynamic linking (or synonymously: dynamic loading) of code is
possible, for \App{} executables compiled with dynamic-linking support (see
Section~\fullref{sec:finding-out-details} on how to check this feature), \App{} can work with
user-defined exposure weighting functions, passed with the long form of
option~\option{--exposure-weight-function}, load the exposure weight function identified by
\metavar{SYMBOL} from \metavar{SHARED-OBJECT} and optionally pass some \metavar{ARGUMENT}s:

\begin{literal}
  --exposure-weight-function=\metavar{SHARED-OBJECT}:\feasiblebreak
  \metavar{SYMBOL}\optional{:\feasiblebreak
    \metavar{ARGUMENT}\optional{:\dots}}%
  \optidx{--exposure-weight-function}
\end{literal}

Some notes on the arguments of this option:

\begin{itemize}
\item
  \metavar{SHARED-OBJECT} is a filename (typically ending in \filename{.so} or \filename{.dll}).
  Depending on the operating system and the dynamic-loader implementation compiled into \appcmd,
  \metavar{SHARED-OBJECT} may or may not require a path.

  \genidx{environment variable!LD\_LIBRARY\_PATH@\envvar{LD\_LIBRARY\_PATH}}%
  \gensee{LD\_LIBRARY\_PATH@\envvar{LD\_LIBRARY\_PATH}}%
         {environment variable, \envvar{LD\_LIBRARY\_PATH}}%
  \begin{restrictedmaterial}{Linux}
    If \metavar{SHARED-OBJECT} does not contain a slash (\sample{/}), the dynamic loader
    \emph{only} searches along \envvar{LD\_LIBRARY\_PATH}; it even ignores the current working
    directory unless \envvar{LD\_LIBRARY\_PATH} contains a dot (\sample{.}).  So, to use a
    \metavar{SHARED-OBJECT} living in the current directory either say
    \begin{literal}
      env LD\_LIBRARY\_PATH=\$LD\_LIBRARY\_PATH:. \bslash \\
      ~~~~\app{} --exposure-weight-function=\feasiblebreak SHARED-OBJECT:\dots
    \end{literal}
    or
    \begin{literal}
      \app{} --exposure-weight-function=\feasiblebreak ./SHARED-OBJECT:\dots
    \end{literal}

    \genidx{dlopen!Linux}%
    For details of the search algorithm please consult the manual page of \manpage{dlopen}{3}.
  \end{restrictedmaterial}

  \smallskip

  \genidx{environment variable!LD\_LIBRARY\_PATH@\envvar{LD\_LIBRARY\_PATH}}%
  \gensee{LD\_LIBRARY\_PATH@\envvar{LD\_LIBRARY\_PATH}}%
         {environment variable, \envvar{LD\_LIBRARY\_PATH}}%
  \genidx{environment variable!DYLD\_LIBRARY\_PATH@\envvar{DYLD\_LIBRARY\_PATH}}%
  \gensee{DYLD\_LIBRARY\_PATH@\envvar{DYLD\_LIBRARY\_PATH}}%
         {environment variable, \envvar{DYLD\_LIBRARY\_PATH}}%
  \genidx{environment variable!DYLD\_FALLBACK\_LIBRARY\_PATH@\envvar{DYLD\_FALLBACK\_LIBRARY\_PATH}}%
  \gensee{DYLD\_FALLBACK\_LIBRARY\_PATH@\envvar{DYLD\_FALLBACK\_LIBRARY\_PATH}}%
         {environment variable, \envvar{DYLD\_FALLBACK\_LIBRARY\_PATH}}%
  \begin{restrictedmaterial}{OS~X}
    \genidx{file!Mach-O}%
    \gensee{Math-O file}{file, Mach-O}%
    If \metavar{SHARED-OBJECT} does not contain a slash (\sample{/}), the dynamic loader
    searches the following paths or directories until it finds a compatible Mach-O~file:

    \begin{enumerate}
    \item \envvar{LD\_LIBRARY\_PATH},
    \item \envvar{DYLD\_LIBRARY\_PATH},
    \item current working directory, and finally
    \item \envvar{DYLD\_FALLBACK\_LIBRARY\_PATH}.
    \end{enumerate}

    \genidx{dlopen!OS~X}%
    For details of the search algorithm please consult the manual page of \manpage{dlopen}{3}.
  \end{restrictedmaterial}

  \smallskip

  \begin{restrictedmaterial}{Windows}
    If \metavar{SHARED-OBJECT} specifies an absolute filename, exactly this file is used.
    Otherwise \App{} searches in the following directories and in this order:

    \begin{enumerate}
    \item The directory from which \appcmd{} is loaded.
    \item The system directory.
    \item The Windows directory.
    \item The current directory.
    \item The directories that are listed in the \envvar{PATH}~environment variable.
    \end{enumerate}

    \genidx{LoadLibrary (Windows)}%
    For details consult the manual page of LoadLibrary.
  \end{restrictedmaterial}

\item
  There is no way knowing which of the symbols inside of \metavar{SHARED-OB\shyp
    JECT} are suitable for \metavar{SYMBOL} without knowledge of source code of
  \metavar{SHARED-OBJECT}.

\item
  A weight function has access to additional \metavar{ARGUMENT}s passed in by appending them
  after \metavar{SYMBOL} with the usual delimiters.  How these \metavar{ARGUMENT}s are
  interpreted and how many of them are required is encoded in the weight-function.  \App{}
  supports \metavar{ARGUMENT}s; it neither restricts their number nor their type.

  \begin{geeknote}%
    \optidx{--exposure-optimum}%
    \optidx{--exposure-width}%
    Usually, neither the exposure optimum
    (\option{--exposure-optimum}=\feasiblebreak\metavar{OPTIMUM}) nor the width
    (\option{--exposure-width}=\feasiblebreak\metavar{WIDTH}) of the exposure function are
    \metavar{ARGUMENT}s, because they are always explicitly passed on to any exposure weight
    function.
  \end{geeknote}

  For example, assuming \filename{variable\_power.cc} of the supplied examples was compiled to
  \filename{variable\_power.so}, we can override the default exponent of 2 with

  \begin{literal}
    \app{} --exposure-weight-function=\bslash \\
    ~~~~~~~~~variable\_power.so:vpower:1.8 \dots
  \end{literal}
\end{itemize}


\subsubsection[Prerequisites]{\label{sec:prerequisites}%
  \genidx{weighting!exposure!prerequisites}%
  Prerequisites}

To use a home-grown exposure-weight function several prerequisites must be met.  On the software
side

\begin{enumerate}
\item
  The operating system allows loading additional code during the execution of an application.

  \genidx{dynamic linking support}%
\item
  \App{} is compiled with the extra feature ``dynamic linking support''.

\item
  Either

  \begin{enumerate}
  \item
    The same compiler that compiled \App{} is available or at least

  \item
    A compiler that produces compatible object code to the compiler that compiled \App{}.
  \end{enumerate}

  The latter is called ``\acronym{ABI}-compatible''.  An example for a pair of
  \acronym{ABI}-compatible compilers is \acronym{GNU}'s~\command{g++} and
  \propername{Intel's}~\command{icpc}.

  \optidx{--show-software-components}%
  To find out which compiler built \emph{your} version of \appcmd{} use
  option~\option{--show\hyp soft\shyp ware\hyp com\shyp po\shyp
    nents}.

\item
  The base-class header file \filename{exposure\_weight\_base.h} is available.
\end{enumerate}

Between chair and keyboard:

\begin{itemize}
\item
  A firm understanding of weighting pixels in the fusion process and in particular in the
  cumulative ascription of different weights.

\item
  A basic understanding of object-oriented programming paired with the ability to compile and
  link single-source C++-files.

\item
  A realistic expectation of the limitations of tailoring weight functions.
\end{itemize}


\subsubsection[Coding Guidelines]{\label{sec:coding-guidelines}%
  \genidx{coding guidelines}%
  Coding Guidelines}

\begin{enumerate}
\item
  \begin{sloppypar}
    Derive the weight function from the supplied C++ base-class~\code{ExposureWeight}, which is
    defined in header file~\filename{exposure\_weight\_base.h}.  It resides in the
    \filename{src}~subdirectory of the source distribution and -- for a correctly installed
    package -- in directory \filename{\val*{val:DOCDIR}\slash examples\slash enfuse}.
  \end{sloppypar}

\item
  At least override member function~\code{weight}.

  \begin{itemize}
  \item
    Domain: define \code{weight} for normalized luminance values~\metavar{y} from zero to one
    including both interval ends: $0 \le y \le 1$.

  \item
    Image: Let the \code{weight}~$w$ fall in the interval from zero to one: $0 \le w \le 1$.
    The \code{weight}s can be all the same, $w = \mbox{const}$.  This is, they can encode a
    constant weight, as long as the constant is not zero.

    \App{} checks this property and refuses to continue if any weight is outside the required
    range or all weights are zero.

  \item
    (Optionally) Rescale the \metavar{WIDTH} of the function to match the \acronym{FWHM} of
    \App{}'s original Gauss curve.  The macro~\code{FWHM\_GAUSSIAN} is defined exactly to this
    end.
  \end{itemize}

\item
  If necessary, rewrite methods~\code{initialize} and \code{normalize}, too.

\item
  \restrictednote{\acronym{OpenMP}-enabled versions only.}

  \App{} never calls \code{initialize} in an \acronym{OpenMP} parallel execution environment.
  However, \acronym{OpenMP}-enabled versions of \App{} call \code{normalize} and \code{weight}
  in parallel sections.

  Technically, the functors which the user-defined weight functions are part of are
  copy-constructed for each \acronym{OpenMP} worker thread.  Thus, there is no contention within
  the derived classes of \code{ExposureWeight}.  Although, if \code{normalize} or \code{weight}
  access a shared resource these accesses must be protected by serialization instructions.  One
  solution is to use \acronym{OpenMP} directives, like for example,

  \begin{cxxlisting}
#pragma omp critical
{
    std::cout << "foobar!" << std::endl;
}
  \end{cxxlisting}

  Experienced hackers will recognize occasions when to prefer other constructs, like, for
  example \code{\#pragma omp atomic} or simply an atomic data-type (for example
  \code{sig\_atomic\_t} from \filename{signal.h}).

  Remember to compile all modules that use \acronym{OpenMP} directives with the
  (compiler-specific) flags that turn on \acronym{OpenMP}.  For \command{g++} this is
  \sample{-fopenmp} and for \command{icpc} it is \sample{-fopenmp} or \sample{-openmp}.

\item
  To raise an exception associated with the derived, user\hyp defined exposure\hyp weight class,
  throw
  \begin{literal}
    ExposureWeight::error(const std::string\& message)
  \end{literal}
  \App{} catches these exceptions at an enclosing scope, displays \metavar{message}, and
  terminates.

\item
  Define an object of the derived class.  This creates the \metavar{SYMBOL} to refer to at the
  \App{} command line.

  The actual signature of the constructor (default, two-argument, \dots) does not matter,
  because \App{} \emph{always} invokes \code{initialize} before calling any other member
  function of a user-defined, derived class of \code{ExposureWeight}.  Member
  function~\code{initialize} sets (read: overwrites) \metavar{optimum} and \metavar{width} and
  ensures they are within the required parameter range.
\end{enumerate}

\exampleName~\ref{ex:simple-dynamic-exposure-weight-function} shows the C++-code of a suitable
extension.  If \App{} has been compiled with support for user-defined weight functions, the
examples presented here should have been duplicated in
directory~\filename{\val*{val:DOCDIR}\slash examples\slash enfuse} along with a
\acronym{GNU}-Makefile called \filename{Makefile.userweight}.


\begin{exemplar}
  \begin{maxipage}
    \begin{cxxlisting}
#include <cmath>                     // std::fabs()

#include "exposure_weight_base.h"    // FWHM_GAUSSIAN, ExposureWeight

struct Linear : public ExposureWeight {
    void initialize(double y_optimum, double width_parameter,
                    ExposureWeight::argument_const_iterator arguments_begin,
                    ExposureWeight::argument_const_iterator arguments_end)
        override {
        ExposureWeight::initialize(y_optimum,
                                   width_parameter * FWHM_GAUSSIAN,
                                   arguments_begin, arguments_end);
    }

    double weight(double y) override {
        const double z = std::fabs(normalize(y));
        return z <= 1.0 ? 1.0 - z : 0.0;
    }
};

Linear linear;
    \end{cxxlisting}
  \end{maxipage}

  \caption[Simple dynamic exposure weight function]%
          {\label{ex:simple-dynamic-exposure-weight-function}%
            A dynamic exposure weight function that defines a ``roof-top''.  The natural width
            is exactly one, so we override member function~\code{initialize} to rescale
            \metavar{WIDTH}, passed in as \code{width\_parameter}, by multiplying with
            \code{FWHM\_GAUSSIAN} to get the same width as the predefined Gaussian.}
\end{exemplar}


As the extension language is C++, we can write templated families of functions, like
\exampleName~\ref{ex:templated-dynamic-exposure-weight-function} demonstrates.


\begin{exemplar}
  \begin{maxipage}
    \begin{cxxlisting}
#include <algorithm>    // std::max()
#include <cmath>        // M_LN2, std::exp(), std::fabs()

#include "exposure_weight_base.h" // FWHM_GAUSSIAN, ExposureWeight

template <int n> double ipower(double x) {return x * ipower<n - 1>(x);}
template <> double ipower<0>(double) {return 1.0;}

template <int n> struct TemplatedPower : public ExposureWeight {
    void initialize(double y_optimum, double width,
                    ExposureWeight::argument_const_iterator arguments_begin,
                    ExposureWeight::argument_const_iterator arguments_end)
        override {
        const double fwhm = 2.0 / std::exp(M_LN2 / static_cast<double>(n));
        ExposureWeight::initialize(y_optimum,
                                   width * FWHM_GAUSSIAN / fwhm,
                                   arguments_begin, arguments_end);
    }

    double weight(double y) override {
        return std::max(1.0 - ipower<n>(std::fabs(normalize(y))), 0.0);
    }
};

TemplatedPower<2> tpower2;
TemplatedPower<3> tpower3;
TemplatedPower<4> tpower4;
    \end{cxxlisting}
  \end{maxipage}

  \caption[Templated dynamic exposure weight function]%
          {\label{ex:templated-dynamic-exposure-weight-function}%
            The templated class~\code{TemplatedPower} allows to create a weight function for
            arbitrary positive exponents~\code{n}.  In particular, \code{TemplatedPower<4>}
            duplicates the built-in exposure-weight function~\code{bisquare}.}
\end{exemplar}


The last example, \ref{ex:variable-dynamic-exposure-weight-function}, shows a weight function
that accesses an extra \metavar{ARGUMENT} passed in with \option{--exposure-weight-function}.  A
class like \code{VariablePower} allows full control over the exponent at the command line
including fractional exponents thereby generalizing both of the previous examples.


\begin{exemplar}
  \begin{maxipage}
    \begin{cxxlisting}
#include <algorithm>    // std::max()
#include <cerrno>       // errno
#include <cmath>        // M_LN2, std::exp(), std::fabs(), std::pow()

#include "exposure_weight_base.h" // FWHM_GAUSSIAN, ExposureWeight


class VariablePower : public ExposureWeight {
    typedef ExposureWeight super;

public:
    void initialize(double y_optimum, double width,
                    ExposureWeight::argument_const_iterator arguments_begin,
                    ExposureWeight::argument_const_iterator arguments_end)
        override {
        if (arguments_begin == arguments_end) {
            exponent = 2.0;
        } else {
            char* tail;
            errno = 0;
            exponent = strtod(arguments_begin->c_str(), &tail);
            if (*tail != 0 || errno != 0) {
                throw super::error("non-numeric exponent");
            }
            if (exponent <= 0.0 || exponent > 4.0) {
                throw super::error("exponent x out of range 0 < x <= 4");
            }
        }

        const double fwhm = 2.0 / std::exp(M_LN2 / exponent);
        super::initialize(y_optimum, width * FWHM_GAUSSIAN / fwhm,
                          arguments_begin, arguments_end);
    }

    double weight(double y) override {
      return std::max(1.0 - std::pow(std::fabs(normalize(y)), exponent), 0.0);
    }

private:
    double exponent;
};

VariablePower vpower;
    \end{cxxlisting}
  \end{maxipage}

  \caption[Dynamic exposure weight function with extra arguments]%
          {\label{ex:variable-dynamic-exposure-weight-function}%
            Dynamic exposure weight function that accesses the first extra argument from the
            tuple of arguments passed with option~\option{--exposure-weight-function}.}
\end{exemplar}


\subsubsection[Performance Considerations]{\label{sec:performance-considerations}%
  \genidx{performance considerations}%
  Performance Considerations}

Exposure weighting objects are created and destroyed only $O(1)$~times.  Thus, member
function~\code{initialize} could be used to perform all kinds of computationally expensive
tasks.  In contrast, methods~\code{normalize} and \code{weight} are called for \emph{every}
pixel in \emph{each} of the input images.  Therefore, if performance of the weight function is a
problem, these two functions are the prime candidates for optimization.


\subsubsection[Compiling, Linking, and Loading]{\label{sec:compiling-linking-loading}%
  Compiling, Linking, and Loading}

\begin{restrictedmaterial}{Linux}
  \noindent Compile and link using the \uref{\gccgnuorg}{\acronym{GNU}-compiler}, \code{g++},
  for example with

  \begin{literal}
    g++ -std=c++11 \bslash \\
    ~~~~-O2 -fpic -I<PATH-TO-BASE-CLASS-HEADER> \bslash \\
    ~~~~-shared -Wl,-soname,dynexp.so \bslash \\
    ~~~~-o dynexp.so \bslash \\
    ~~~~dynexp.cc
  \end{literal}

  The important options are

  \begin{codelist}
  \item[\option{-fpic}]\itemend
    Instruct the compiler's code-generator to produce position\hyp{}independent code
    (\acronym{PIC}), which is suitable for a shared object.  Some systems require \sample{-fPIC}
    instead of \sample{-fpic}.

  \item[\option{-shared}]\itemend
    Tell the linker to create a shared object instead of the default executable.  On some
    systems, the library must be ``blessed'', by passing the shared-object name
    (\option{soname}) directly to the linker (\option{-Wl}).

    Of course more than one object file can be linked into a single shared object.
  \end{codelist}

  Finally, the weight function can be selected by its \metavar{SYMBOL}~name in the
  \metavar{SHARED-OBJECT}.

  \begin{literal}
    \app{} --exposure-weight-function=dynexp.so:linear\dots
  \end{literal}
\end{restrictedmaterial}

\medskip

\begin{restrictedmaterial}{OS~X}
  \noindent On OS~X the creation of shared objects -- or loadable modules -- has been tested
  with the C-language frontend of \uref{\llvm}{\acronym{LLVM}},
  \uref{\clangllvm}{\code{clang++}}, and should work on OS~X Mavericks~(10.9) or higher.

  \begin{literal}
    clang++ -std=c++11 -stdlib=libc++ \bslash \\
    ~~~~-O2 -bundle -I<PATH-TO-BASE-CLASS-HEADER> \bslash \\
    ~~~~-o dynexp.so \bslash \\
    ~~~~dynexp.cc
  \end{literal}

  The important option here is \sample{-bundle} which instructs the compiler's code-generator to
  produce a loadable module.

  Finally, the weight function can be selected by its \metavar{SYMBOL}~name in the
  \metavar{SHARED-OBJECT}.

  \begin{literal}
    \app{} --exposure-weight-function=dynexp.so:linear\dots
  \end{literal}
\end{restrictedmaterial}

\medskip

\begin{restrictedmaterial}{Windows}
  \noindent On Windows the creation of shared objects -- or dynamic link libraries
  (\acronym{DLL}~files) as they are called here -- has been tested with the \acronym{MinGW}
  compiler chain and with MS-Visual~C++~2012.

  \begin{itemize}
  \item
    Compile and link using the \acronym{MinGW} compiler with

    \begin{literal}
      g++ -g -O2 -I<PATH-TO-BASE-CLASS-HEADER> -c dynexp.cc \\
      g++ -g -shared -Wl,-soname,dynexp.dll -o dynexp.dll dynexp.o
    \end{literal}

    For details see the explanation for the \acronym{GNU} compiler above.  Windows neither
    requires options \option{-fpic} nor~\option{-fPIC}.

  \item
    When using the MS-Visual~C++~compiler, you need to explicitly export \metavar{SYMBOL}.
    There are two possibilities to achieve this.  Use only one variant, not both at the same
    time.

    \begin{enumerate}
    \item
      Either use \code{"C"}~linkage and define the object using the construction
      \code{\_\_declspec(dllexport)}.  For
      \exampleName~\ref{ex:simple-dynamic-exposure-weight-function} the object definition has to
      be extended to

      \begin{cxxlisting}
extern "C"
{
  __declspec(dllexport) Linear linear;
}
      \end{cxxlisting}

    \item
      Or, alternatively, create a module-definition file (\filename{.def}) and pass this file to
      the linker (in: \guielement{Project Properties}, \guielement{Linker}, \guielement{Module
        Definition File}).  For \exampleName~\ref{ex:simple-dynamic-exposure-weight-function},
      this file would look like

      \begin{literal}
        LIBRARY dynexp \\
        EXPORTS \\
        ~~~~~~~~linear @@1 \\
      \end{literal}
    \end{enumerate}
  \end{itemize}

  Finally, the weight function can be selected by its \metavar{SYMBOL} in the dynamic link
  library.

  \begin{literal}
    \app{} --exposure-weight-function=dynexp.dll:linear\dots
  \end{literal}
\end{restrictedmaterial}

\begin{optionsummary}
\item[--exposure-optimum] Section~\fullref{opt:exposure-optimum}
\item[--exposure-weight-function] Section~\fullref{opt:exposure-weight-function}
\item[--exposure-weight] Section~\fullref{opt:exposure-weight}
\item[--exposure-width] Section~\fullref{opt:exposure-width}
\item[--gray-projector] Section~\fullref{opt:gray-projector}
\end{optionsummary}


%%% Local Variables:
%%% fill-column: 96
%%% End:

%% This file is part of Enblend.
%% Licence details can be found in the file COPYING.


\subsection[User-Defined \acronym{OpenCL} Functions]{\label{sec:user-defined-opencl-functions}%
  \genidx{weighting!exposure!user-defined!\acronym{OpenCL}}%
  User-Defined \acronym{OpenCL} Exposure Weighting Functions}

See also \sectionName~\ref{sec:user-defined-dynlink-functions} above on defining C++-weighting
functions via dynamic linking.


\subsubsection[Introduction]{\label{sec:user-defined-opencl-functions-introduction}%
  \genidx{weighting!exposure!\acronym{OpenCL} introduction}%
  Introduction}

\genidx{\acronym{GPGPU}}%
\genidx{\acronym{OpenCL}}%

So what has \acronym{OpenCL} to do with user-defined exposure weighting functions?  Every
\acronym{OpenCL} system working on top of a \acronym{GPGPU} comes with a compiler and a linker
specifically tailored for the \acronym{GPGPU}.  Both tools are hidden in one or more libraries
of the respective graphics card.  \App{} reaps the benefits from these libraries already having
been installed and uses the \acronym{OpenCL}-host interface to compile, link and execute
\acronym{OpenCL}-programs on the \acronym{GPGPU}.

In brief, if \appcmd{} has been built with \acronym{OpenCL} support, the system has all
necessary libraries installed, the user can feed \acronym{OpenCL}-source code -- actually a
C-dialect -- to \App{} without the need to install and maintain all the tools to compile
\appcmd{} itself as in the previously described method of dynamic linking.

The ability of \acronym{GPGPU}s to crunch humongous amounts of data is unimportant here.

\noindent\restrictednote{\acronym{OpenCL}-enabled versions only.}

\optidx{--exposure-weight-function}% \optidx{--gpu}% For \App{} executables compiled with
\acronym{OpenCL} support (see Section~\fullref{sec:finding-out-details} on how to check this
feature), \App{} can work with user-defined exposure weighting functions, passed with the long
form of option~\option{--exposure-weight-function} after \acronym{GPU}-support has been
requested with option~\option{--gpu}.  \App{} compiles, links and evaluables the exposure weight
function identifed by \metavar{SYMBOL} from \metavar{OPENCL-SOURCE}, where
\metavar{OPENCL-SOURCE} has extension~\filename{.cl}:

\begin{literal}
  --gpu --exposure-weight-function=\metavar{OPENCL-SOURCE}:\feasiblebreak
  \metavar{SYMBOL}\optional{:\feasiblebreak
    \metavar{ARGUMENT}\optional{:\dots}}%
  \optidx{--exposure-weight-function}
\end{literal}

If \metavar{SYMBOL} is not found in \metavar{OPENCL-SOURCE} or does not define a suitable
function, \app{} aborts with an extensive error message.  In
\exampleName~\ref{ex:opencl-userweight-error} we supply a wrong \metavar{SYMBOL} called
\sample{foobar}, whereas \filename{variable\_power.cl} -- one of the supplied examples --
actually defines \sample{weight}.

\begin{exemplar}[htbp]
  \begin{maxipage}
    \centering
    \begin{terminal}
\$ enfuse --gpu --exposure-weight-function=variable\_power.cl:foobar image-[0-9].tif \\
enfuse: info: chose OpenCL platform \#2, NVIDIA Corporation, NVIDIA CUDA, device \#1 \\
enfuse: invalid binary \\
enfuse: note: build status \\
enfuse: note:~~~~~error (status-code = -2) \\
enfuse: note: build options \\
enfuse: note: ~~~~-cl-nv-verbose \bslash \\
~~~~~~~~~~~~~~~~~~-cl-single-precision-constant \bslash \\
~~~~~~~~~~~~~~~~~~-I/usr/local/share/enblend/kernels  \bslash \\
~~~~~~~~~~~~~~~~~~-I/usr/share/enblend/kernels \bslash \\
~~~~~~~~~~~~~~~~~~-DENFUSE\_FWHM\_GAUSSIAN=2.354820f \bslash \\
~~~~~~~~~~~~~~~~~~-DENFUSE\_OPTIMUM\_Y=0.5f \bslash \\
~~~~~~~~~~~~~~~~~~-DENFUSE\_USER\_WEIGHT\_FUNCTION=foobar \bslash \\
~~~~~~~~~~~~~~~~~~-DENFUSE\_WIDTH=0.2f \\
enfuse: note: build log \\
enfuse: note:~~~~~ptxas fatal: Unresolved extern function 'foobar'
    \end{terminal}
  \end{maxipage}

  \caption[\acronym{OpenCL} uer-weight error]{%
    \label{ex:opencl-userweight-error}%
    Error output of \App{} for an invalid (here: unknown) user-defined \acronym{OpenCL}
    exposure-weight function.}
\end{exemplar}


\subsubsection[\acronym{OpenCL} Coding Guidelines]{\label{sec:opencl-coding-guidelines}%
  \genidx{coding guidelines!\acronym{OpenCL}}%
  \acronym{OpenCL} Coding Guidelines}

\begin{enumerate}
\item
  Define a weight-function with following signature
  \begin{cxxlisting}
float weight(float y);
  \end{cxxlisting}
  in a text file.  It takes a luminance in the normalized interval~$(0, 1)$ and returns the
  weight.  This function name will become \metavar{SYMBOL} at the \App{} command line.

\item
  Pass filename~\metavar{OPENCL-SOURCE} (with extension~\filename{.cl}) of the text file and the
  function name~\metavar{SYMBOL} to \App{} as shown above.

  \genidx{environment variable!ENBLEND\_OPENCL\_PATH@\envvar{ENBLEND\_OPENCL\_PATH}}%
\item
  \metavar{OPENCL-SOURCE} can include (\code{\#include "\dots"}) other files.  The include path
  is \envvar{ENBLEND\_OPENCL\_PATH}.  Note that the environment variable starts with
  ``ENBLEND'', not ``ENFUSE''.

\item
  All \metavar{ARGUMENT}s are converted to preprocessor symbols (\code{\#define \dots}) and
  passed to the \acronym{OpenCL}-compiler when it processes \metavar{OPENCL-SOURCE}.  For
  example, adding an \metavar{ARGUMENT} like \sample{EXPONENT=1.25} will add
  \sample{-DEXPONENT=1.25} to the compiler options, which is equivalent to saying
  \begin{cxxlisting}
#define EXPONENT 1.25
  \end{cxxlisting}
  at the top of \metavar{OPENCL-SOURCE}.

  To make the \metavar{ARGUMENT}s optional at the \app{} command line use the
  \begin{cxxlisting}
#ifndef ARGUMENT
#define ARGUMENT ...
#endif
  \end{cxxlisting}
  idiom in the source code and supply a reasonable default.

  \genidx{OpenCL helpers@\acronym{OpenCL} helpers}
\item
  \App{} supplies a set of preprocessor symbols and functions to each user-defined function.
  They ease writing weight functions.  Everyone starts with \sample{ENFUSE\_} (preprocessor
  symbols) or \sample{enfuse\_} (functions) to avoid name collisions and shadowing.

  \begin{codelist}
    \genidx{OpenCL helpers@\acronym{OpenCL} helpers!ENFUSE-OPTIMUM-Y@\code{ENFUSE\_OPTIMUM\_Y}}%
    \gensee{ENFUSE-OPTIMUM-Y@\code{ENFUSE\_OPTIMUM\_Y}}{\acronym{OpenCL} helpers, \code{ENFUSE\_OPTIMUM\_Y}}%
    \genidx{exposure!optimum}%
    \optidx{--exposure-optimum}%
  \item[ENFUSE\_OPTIMUM\_Y]\itemend The value of the optimum luminance in the normalized
    luminance interval.  The user sets this value with option~\option{--exposure-optimum}
    (default:~\val{val:minimum-exposure-optimum}, see \fullref{opt:exposure-optimum}).

    \genidx{OpenCL helpers@\acronym{OpenCL} helpers!ENFUSE-WIDTH@\code{ENFUSE\_WIDTH}}%
    \gensee{ENFUSE-WIDTH@\code{ENFUSE\_WIDTH}}{\acronym{OpenCL} helpers, \code{ENFUSE\_WIDTH}}%
    \genidx{exposure!weight}%
    \optidx{--exposure-width}%
  \item[ENFUSE\_WIDTH]\itemend The characteristic width of the weight function.  Set this value
    at the command line with option~\option{--exposure-width}
    (default:~\val{val:default-exposure-width}, see \fullref{opt:exposure-width}).

    \genidx{OpenCL helpers@\acronym{OpenCL} helpers!ENFUSE-FWHM-GAUSSIAN@\code{ENFUSE\_FWHM\_GAUSSIAN}}%
    \gensee{ENFUSE-FWHM-GAUSSIAN@\code{ENFUSE\_FWHM\_GAUSSIAN}}%
           {\acronym{OpenCL} helpers, \code{ENFUSE\_FWHM\_GAUSSIAN}}%
  \item[ENFUSE\_FWHM\_GAUSSIAN]\itemend Shorthand for the width of \App's first ever weight
    function.  Use the value to rescale the width of the newly written user function to make it
    bahave like a predefined one with respect to the width.

    The actual value, $2 \sqrt{2 \log(2)}$, is approximately 2.35482.

    \genidx{OpenCL helpers@\acronym{OpenCL} helpers!enfuse-normalized-luminance@\code{enfuse\_normalized\_luminance}}%
    \gensee{enfuse-normalized-luminance@\code{enfuse\_normalized\_luminance}}%
           {\acronym{OpenCL} helpers, \code{enfuse\_normalized\_luminance}}%
  \item[enfuse\_normalized\_luminance]\itemend Helper function which implements
    \equationabbr~\ref{equ:linear-luminance-transform}.
    \begin{cxxlisting}
float enfuse_normalized_luminance(float y)
{
    return (y - ENFUSE_OPTIMUM_Y) / ENFUSE_WIDTH;
}
    \end{cxxlisting}
  \end{codelist}
\end{enumerate}

\begin{geeknote}
  The weight function's signature being so simple and weight functions usually also being simple
  none of the language-specific features like, for example, \code{local}-\slash\code{global}
  address spaces, synchronization or vector-data types are requied to write useful weight
  functions.
\end{geeknote}

Examples~\ref{ex:opencl-generalized-gaussian-weight} and
\ref{ex:variable-opencl-exposure-weight-function} show how weight functions can be constructed
using \acronym{OpenCL}.  Example~\ref{ex:opencl-generalized-gaussian-weight} implements
\[
    w(z) = \exp\left(-|z|^{\mathtt{EXPONENT}}\right),
\]
where $\code{EXPONENT} = 2$ obviously duplicates the pre-defined \propername{Gaussian} weight,
\equationabbr~\ref{equ:weight:gauss}.

\begin{exemplar}[htbp]
  \begin{cxxlisting}
#ifndef EXPONENT
#define EXPONENT 2.0f
#endif

float weight(float y)
{
    return exp(-pow(fabs(enfuse_normalized_luminance(y)),
                    EXPONENT));
}
  \end{cxxlisting}

  \caption[Generalized \propername{Gauss} weight function]{%
    \label{ex:opencl-generalized-gaussian-weight}%
    Generalized \propername{Gauss} weight function written in \acronym{OpenCL}.  Note the
    definition of a default parameter, \code{EXPONENT}.}
\end{exemplar}

Say \exampleName~\ref{ex:opencl-generalized-gaussian-weight} is stored in
\filename{generalized\_gaussian.cl} and reachable via an element in
\envvar{ENBLEND\_OPENCL\_PATH} then is can be used as
\begin{literal}
  \app{} --gpu --exposure-weight-function=\feasiblebreak
  generalized\_gaussian.cl:\feasiblebreak
  weight \dots
\end{literal}
or
\begin{literal}
  \app{} --gpu --exposure-weight-function=\feasiblebreak
  generalized\_gaussian.cl:\feasiblebreak
  weight:\feasiblebreak
  1.5 \dots
\end{literal}

\exampleName~\ref{ex:variable-opencl-exposure-weight-function} implements the same variable
weight as \exampleName~\ref{ex:variable-dynamic-exposure-weight-function} without the parameter
checks:
\[
    w(z) =
    \left\{
      \begin{array}{cl}
        1 - |z|^{\mathtt{EXPONENT}} & \mbox{if } |z| \leq 1 \\
        0                         & \mbox{otherwise.}
      \end{array}
    \right.
\]

\begin{exemplar}[htbp]
  \begin{cxxlisting}
#ifndef M_LN2
#define M_LN2 0.69314718f
#endif

#ifndef EXPONENT
#define EXPONENT 2.0f
#endif

float normalized_luminance(float y)
{
    const float fwhm = 2.0f / exp(M_LN2 / EXPONENT);

    return enfuse_normalized_luminance(y) *
           (fwhm / ENFUSE_FWHM_GAUSSIAN);
}

float variable_power_weight(float y)
{
    return max(1.0f - pow(fabs(normalized_luminance(y)),
                          EXPONENT),
               0.0f);
}
  \end{cxxlisting}

  \caption[\acronym{OpenCL} exposure weight function with an extra argument]{%
    \label{ex:variable-opencl-exposure-weight-function}%
    \acronym{OpenCL} exposure weight function with extra argument~\code{EXPONENT} that defaults
    to 2.  Also compare with \exampleName~\ref{ex:variable-dynamic-exposure-weight-function} for
    the implementation of almost the same functionality using dynamic linking.}
\end{exemplar}

Say \exampleName~\ref{ex:variable-opencl-exposure-weight-function} is stored in
\filename{variable\_power.cl} and reachable via an element in
\envvar{ENBLEND\_OPENCL\_PATH} then is can be used as
\begin{literal}
  \app{} --gpu --exposure-weight-function=\feasiblebreak
  variable\_power.cl:\feasiblebreak
  variable\_power\_weight \dots
\end{literal}
or
\begin{literal}
  \app{} --gpu --exposure-weight-function=\feasiblebreak
  variable\_power.cl:\feasiblebreak
  variable\_power\_weight:\feasiblebreak
  3 \dots
\end{literal}

\optidx{--exposure-weight}%
\optidx{--exposure-optimum}%
\optidx{--exposure-width}%
The options~\option{--exposure-weight}, \option{--exposure-optimum} and
\option{--exposure-width} work as expected for user-defined functions, too.

\begin{sgquote}
  \Quote{All debts have now been paid.}  --
  \Author{\genidx{Linea@\propername{Linea}}%
  \gensee{Ke'ra@\propername{Ke'ra}}{\propername{Linea}}%
  \propername{Linea}, Destroyer of Worlds (\propername{Ke'ra})}
\end{sgquote}


%%% Local Variables:
%%% fill-column: 96
%%% End:



%%% Local Variables:
%%% fill-column: 96
%%% End:
