%% This file is part of Enblend.
%% Licence details can be found in the file COPYING.


\subsection[Mask Generation Options]{\label{sec:mask-generation-options}%
  \genidx[\rangebeginlocation]{mask generation options}%
  \genidx{options!mask generation}%
  Mask Generation Options}

These options control the generation and the usage of masks.

\begin{codelist}
  \label{opt:coarse-mask}%
  \optidx[\defininglocation]{--coarse-mask}%
  \genidx{mask!coarse}
  \gensee{coarse mask}{mask, coarse}%
\item[--coarse-mask\optional{=\metavar{FACTOR}}]\itemend
  Use a scaled-down or ``coarse'' version of the input images to create the seam line.  This
  option reduces the number of computations necessary to compute the seam line and the amount of
  memory necessary to do so.  It is the default.

  If omitted \metavar{FACTOR} defaults to \val{val:default-coarseness-factor}, this means,
  option~\option{--coarse\hyp mask} shrinks the overlapping \emph{areas} by a factor of
  $\val{val:default-coarseness-factor}\classictimes\val{val:default-coarseness-factor}$.  With
  $\metavar{FACTOR} = 8$ the total memory allocated during a run of \App{} shrinks approximately
  by 80\% and the maximum amount of memory in use at a time is decreased to some 40\% in
  comparison to a full-size (``fine'') mask.

  Valid range: $\metavar{FACTOR} = 1, 2, 3, \dots$, where 1 reproduces \code{--fine\hyp mask}.

  Also see the negated option, \flexipageref{\sample{--fine-mask}}{opt:fine-mask} and
  \tableName~\ref{tab:mask-generation}.

  \begin{table}
    \begin{tabular}{lp{.333\linewidth}p{.333\linewidth}}
      \hline
      \multicolumn{1}{c|}{Active Options} &
      \multicolumn{1}{c|}{\option{--no-optimize}} &
      \multicolumn{1}{c}{\option{--optimize}} \\
      \hline\extraheadingsep
      \option{--fine-mask} & Use \acronym{NFT} mask. & Vectorize \acronym{NFT} mask, optimize
      vertices with simulated annealing and \propername{Dijkstra's} shortest path algorithm,
      fill vector contours to recover mask. \\

      \option{--coarse-mask} & Scale down overlap region, compute \acronym{NFT} mask and
      vectorize it, fill vector contours. & Scale down overlap region, vectorize \acronym{NFT}
      mask, optimize vertices with simulated annealing and \propername{Dijkstra's} shortest path
      algorithm, fill vector contours to recover mask.
    \end{tabular}

    \caption[Mask generation options]{\label{tab:mask-generation}%
      \genidx{nearest-feature transform (\acronym{NFT})}%
      \genidx[\summarylocation]{mask!generation}%
      Various options that control the generation of masks.  All mask computations are based on
      the Nearest-Feature Transformation (\acronym{NFT}) of the overlap region.}
  \end{table}


  \label{opt:fine-mask}%
  \optidx[\defininglocation]{--fine-mask}%
  \genidx{mask!fine}
  \gensee{fine mask}{mask, fine}%
\item[--fine-mask]\itemend
  Instruct \App{} to employ the full-size images to create the seam line, which can be slow.
  Use this option, for example, if you have very narrow overlap regions.

  Also see option~\flexipageref{\option{--coarse-mask}}{opt:coarse-mask} and
  \tableName~\ref{tab:mask-generation}.


  \label{opt:load-masks}%
  \optidx[\defininglocation]{--load-masks}%
  \genidx{mask!load}%
  \gensee{load mask}{mask, load}%
\item[--load-masks\optional{=\metavar{IMAGE-TEMPLATE}}]\itemend
  Instead of generating masks, load those in \metavar{IMAGE\hyp{}TEMPLATE}.  The default is
  \mbox{\sample{\val{val:default-mask-template}}}.  The mask images must be 8~bit grayscale
  images.

  See option~\flexipageref{\option{--save-masks}}{opt:save-masks} below for details.


  \label{opt:optimize}%
  \optidx[\defininglocation]{--optimize}%
  \genidx{optimize!seam}%
  \gensee{seam optimization}{optimize, seam}%
\item[--optimize]\itemend
  Use a multi-strategy approach to route the seam line around mismatches in the overlap region.
  This is the default.  \tableName~\ref{tab:optimizer-strategies} explains these strategies;
  also see Table~\ref{tab:mask-generation}.

  \begin{table}
    \begin{tabular}{p{.35\linewidth}p{.55\linewidth}}
      \hline
      \multicolumn{1}{c|}{Algorithm} &
      \multicolumn{1}{c}{Tuning Parameters} \\
      \hline\extraheadingsep
      Simulated Annealing & Tune with option~\option{--anneal} = \metavar{TAU} :
      \metavar{DELTA-E-MAX} : \metavar{DELTA-E-MIN} : \metavar{K-MAX}.
      \uref{\wikipediasimulatedannealing}{Simulated-Annealing}.  \fixme{Explain
        Simulated-Annealing!} \\
      \propername{Dijkstra} Shortest Path & Tune with option~\option{--dijkstra} =
      \metavar{RADIUS}.  \uref{\wikipediadijkstraalgorithm}{\propername{Dijkstra} algorithm}.
      \fixme{Explain \propername{Dijkstra} algorithm!}
    \end{tabular}

    \caption[Optimizer strategies]{\label{tab:optimizer-strategies}%
      \genidx[\summarylocation]{optimize!strategy}%
      \App{}'s strategies to optimize the seam lines between overlapping images.}
  \end{table}

  \genidx{optimize!seam}%
  \gensee{seam optimization}{optimize, seam}%
  \genidx{nearest-feature transform (\acronym{NFT})}%
  Option~\option{--no-optimize}\optidx[\defininglocation]{--no-optimize} negates
  \option{--optimize} and thus turns off seam line optimization.  Combined with
  option~\flexipageref{\option{--fine-mask}}{opt:fine-mask} this will produce the same type of
  mask as \App{}~version~2.5, namely the result of a Nearest-Feature Transform (\acronym{NFT}).


  \label{opt:save-masks}%
  \optidx[\defininglocation]{--save-masks}%
  \genidx{mask!save}%
  \gensee{save mask}{mask, save}%
\item[--save-masks\optional{=\metavar{IMAGE-TEMPLATE}}]\itemend
  Save the generated masks to \metavar{IMAGE-TEMPLATE}, which defaults to
  \mbox{\sample{\val{val:default-mask-template}}}.  \App{} saves masks as 8~bit grayscale,
  i.e.\ single channel images.  For accuracy we recommend to choose a lossless format.

  Use this option if you wish to edit the location of the seam line by hand.  This will give you
  images of the right sizes that you can edit to make your changes.  Later, use
  option~\flexipageref{\option{--load-masks}}{opt:load-masks} to blend the project with your
  custom seam lines.

  \genidx{save mask!only}%
  \gensee{only save mask}{save mask, only}%
  \begin{sloppypar}
    \App{} will stop after saving all masks unless
    option~\flexipageref{\option{--output}}{opt:output} is given, too.  With both options given,
    this is, \sample{--save-masks} and \sample{--output}, \App{} saves all masks and then
    proceeds to blend the output image.
  \end{sloppypar}

  \begin{geeknote}
    \metavar{IMAGE-TEMPLATE} defines a template that is expanded for each input file.  In a
    template a percent sign (\sample{\%}) introduces a variable part.  All other characters are
    copied literally.  Lowercase letters refer to the name of the respective input file, whereas
    uppercase ones refer to the name of the output file (see
    Section~\fullref{sec:common-options}).  \tableName~\ref{tab:mask-template-characters} lists
    all variables.

    A fancy mask filename template could look like this:

    \begin{literal}
      \%D/mask-\%02n-\%f.tif
    \end{literal}

    It puts the mask files into the same directory as the output file~\sample{\%D}, generates a
    two-digit index~\sample{\%02n} to keep the mask files nicely sorted, and decorates the mask
    filename with the name of the associated input file~\sample{\%f} for easy recognition.
  \end{geeknote}


  \label{opt:visualize}%
  \optidx[\defininglocation]{--visualize}%
  \genidx{mask!optimization visualization}%
  \gensee{visualization of mask}{mask, optimization visualization}%
\item[--visualize\optional{=\metavar{VISUALIZE-TEMPLATE}}]\itemend
  Create an image according to \metavar{VISUALIZE-TEMPLATE} that visualizes the un-optimized
  mask and the applied optimizations (if any).  The default is
  \mbox{\sample{\val{val:default-visualize-template}}}.

  This image will show \App{}'s view of the overlap region and how it decided to route the seam
  line.  If you are experiencing artifacts or unexpected output, it may be useful to include
  this visualization image in your bug report.  For a detailed description of the image, consult
  \chapterName~\fullref{sec:visualization-image}.

  \begin{geeknote}
    \metavar{VISUALIZE-TEMPLATE} defines a template that is expanded for each input file.  In a
    template, a percent sign (\sample{\%}) introduces a variable part; all other characters are
    copied literally.  Lowercase letters refer to the name of the respective input file, whereas
    uppercase ones refer to the name of the output file (see
    option~\flexipageref{\option{--output}}{opt:output}).
    \tableName~\fullref{tab:mask-template-characters} lists all variables.
  \end{geeknote}
\end{codelist}


%% This file is part of Enblend.
%% Licence details can be found in the file COPYING.


\begin{table}[htbp]
  \centering
  \begin{tabular}{cp{.8\linewidth}}
    \hline
    \multicolumn{1}{c|}{Format} & \multicolumn{1}{c}{Interpretation} \\
    \hline\extraheadingsep
    \genidx{mask!template character!\%@\sample{\%}}\code{\%\%} &
    Produces a literal \sample{\%}-sign. \\


    \genidx{mask!template character!i@\sample{i}}\code{\%i} & Expands to the index of the mask
    file starting at zero.  \sample{\%i} allows for setting a pad character or a width
    specification:

    \begin{literal}
      \% \metavar{PAD} \metavar{WIDTH} i
    \end{literal}

    \metavar{PAD} is either \sample{0} or any punctuation character; the default pad character
    is \sample{0}.  \metavar{WIDTH} is an integer specifying the minimum width of the number.
    The default is the smallest width given the number of input images, this is 1 for
    2--9~images, 2 for 10--99~images, 3 for 100--999~images, and so on.

    Examples: \sample{\%i}, \sample{\%02i}, or \sample{\%\_4i}. \\


    \genidx{mask!template character!n@\sample{n}}\code{\%n} & Expands to the number of the mask
    file starting at one. Otherwise it behaves identically to \sample{\%i}, including pad
    character and width specification. \\


    \genidx{mask!template character!p@\sample{p}}\code{\%p} & This is the full name (path,
    filename, and extension) of the input file associated with the mask.

    Example: If the input file is called \filename{/home/luser/snap/img.jpg}, \sample{\%p}
    expands to \filename{/home/luser/snap/img.jpg}, or shorter: \sample{\%p} \result{}
    \filename{/home/luser/snap/img.jpg}. \\


    \genidx{mask!template character!P@\sample{P}}\code{\%P} & This is the full name of the
    output file. \\


    \genidx{mask!template character!d@\sample{d}}\code{\%d} & Is replaced with the directory
    part of the associated input file.

    Example (cont.): \sample{\%d} \result{} \filename{/home/luser/snap}. \\


    \genidx{mask!template character!D@\sample{D}}\code{\%D} & Is replaced with the directory
    part of the output file. \\


    \genidx{mask!template character!b@\sample{b}}\code{\%b} & Is replaced with the non-directory
    part (often called ``basename'') of the associated input file.

    Example (cont.): \sample{\%b} \result{} \filename{img.jpg}. \\


    \genidx{mask!template character!B@\sample{B}}\code{\%B} & Is replaced with the non-directory
    part of the output file. \\


    \genidx{mask!template character!f@\sample{f}}\code{\%f} & Is replaced with the filename
    without path and extension of the associated input file.

    Example (cont.): \sample{\%f} \result{} \filename{img}. \\


    \genidx{mask!template character!F@\sample{F}}\code{\%F} & Is replaced with the filename
    without path and extension of the output file. \\


    \genidx{mask!template character!e@\sample{e}}\code{\%e} & Is replaced with the extension
    (including the leading dot) of the associated input file.

    Example (cont.): \sample{\%e} \result{} \filename{.jpg}. \\


    \genidx{mask!template character!E@\sample{E}}\code{\%E} & Is replaced with the extension of
    the output file.
  \end{tabular}

  \caption[Mask template characters]{\label{tab:mask-template-characters}%
    \genidx[\summarylocation]{mask!template character}%
    Special format characters to control the generation of mask filenames.  Uppercase letters
    refer to the output filename and lowercase ones to the input files.}
\end{table}


%%% Local Variables:
%%% fill-column: 96
%%% End:



\genidx[\rangeendlocation]{mask generation options}


%%% Local Variables:
%%% fill-column: 96
%%% End:
