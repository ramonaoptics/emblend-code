%% This file is part of Enblend.
%% Licence details can be found in the file COPYING.


%% Words and abbreviations

\newcommand*{\appendixabbr}{App.}
\newcommand*{\appendixName}{Appendix}
\newcommand*{\chapterabbr}{Ch.}
\newcommand*{\chapterName}{Chapter}
\newcommand*{\equatioabbr}{Eqn.}
\newcommand*{\equationname}{Equation}
\newcommand*{\exampleabbr}{Ex.}
\newcommand*{\examplename}{Example}
\newcommand*{\pageabbr}{p.}
\newcommand*{\pagename}{page}
\newcommand*{\sectionabbr}{Sec.}
\newcommand*{\sectionname}{Section}


%% Texinfo compatibility macros

\newcommand*{\acronym}[1]{\textsc{#1}}
\newcommand*{\code}[1]{\texttt{#1}}
\newcommand*{\command}[1]% command name in running text; use \code everywhere else
{\ifhevea\textbf{#1}\else{\fontseries{b}\selectfont #1}\fi}
\newcommand*{\dmn}[1]{\,#1}
\newcommand*{\envvar}[1]{\texttt{#1}}     % Texinfo: `@env'
\newcommand*{\filename}[1]{\textit{#1}}   % Texinfo: `@file'
\newcommand*{\metavar}[1]{\textsl{#1}}    % Texinfo: `@var'
\newcommand{\option}[1]{\texttt{%
    %% We want to make long-name options (implicitly) hyphenatable in
    %% the printable document, however the initial two dashes must
    %% never be hyphenation points.
    \ifhevea #1\else
    \IfBeginWith{#1}{--}%
                {\mbox{--}\begin{makehyphenatable}{\ttfamily}\StrGobbleLeft{#1}{2}\end{makehyphenatable}}%
                {#1}\fi
}}
\WithSuffix\newcommand\option*[1]{\mbox{\texttt{#1}}} % Texinfo: `@option'
\newcommand*{\sample}[1]{\mbox{`\texttt{#1}'}} % Texinfo: `@samp'
\newcommand*{\strong}[1]{\textbf{#1}}


\newcommand*{\abbreviation}[2][\relax]{\ifx\relax#1{#2}\else{#2~(#1)}\fi}


\newif\ifinfootnote
\ifhevea\relax\else
  \let\oldfootnote\footnote
  \renewcommand{\footnote}[2]{\infootnotetrue\oldfootnote{#1}{#2}\infootnotefalse}
\fi

\newcommand*{\uref}[2]% Texinfo: `@uref{URL, LABEL}'
{\ifhevea
   \footahref{#1}{#2}% See: http://hevea.inria.fr/doc/manual018.html#sec65
 \else
   \ifinfootnote
     \ahref{#1}{#2} (#1)% Texinfo's clumsy style
   \else
     \footahref{#1}{#2}% Hevea's preferred style
   \fi
 \fi}

\newcommand*{\urefval}[2]%
{\ifhevea
   \indicatesourcevalue{\uref{\val*{#1}}{#2}}%
 \else
   \uref{\indicatesourcevalue{\val*{#1}}}{#2}%
 \fi}


%% Use `\urlmark{URL}{LABEL}' inside an `\item' of a `description'
%% list, i.e. inside the square brackets.  Follow with `\urltext'
%% outside the right square bracket.  Example:
%%         \item[\urlmark{\urloffoo}{Foo}]\urltext
\newcommand{\urlmark}[2]%
{\ifhevea
   \footahref{#1}{#2}%
 \else
   {#2}\footnotemark\global\def\urlmarkargi{#1}\fi}

\newcommand{\urltext}%
{\ifhevea\relax
 \else\footnotetext{\texttt{\urlmarkargi}}\fi}


%% Our own markup

\newcommand*{\Ex}{\textsf{Ex}\,}   % expectation value
\newcommand*{\Var}{\textsf{Var}\,} % variance

\newcommand*{\angulardegree}{\mbox{$^{\circ}$}}
\newcommand*{\asterisk}{\mbox{$*$}}
\newcommand*{\atsign}{\char64}
\newcommand*{\bslash}{\char92}
\newcommand*{\caret}{\char94}
\newcommand*{\classictimes}{\mbox{$\times$}}
\newcommand*{\commonpart}{\mbox{$^{\textsf{c}}$}} % Chapters, sections, etc. occurring on both manuals
\newcommand*{\extraheadingsep}{\ifhevea\relax\else\rule{0pt}{3ex}\fi} % was: \rigidbaselineskip
\newcommand*{\feasiblebreak}{\ifhevea\relax\else\discretionary{}{}{}\fi}
\newcommand*{\itemend}{\ifhevea \relax\else \leavevmode\newline \fi}
\newcommand*{\oldstylefirst}{\mbox{$1^{\mathrm{st}}$}}
\newcommand*{\oldstylesecond}{\mbox{$2^{\mathrm{nd}}$}}
\newcommand*{\oldstylethird}{\mbox{$3^{\mathrm{rd}}$}}
\newcommand*{\raisedasterisk}{\mbox{${}^{*}$}}
\newcommand*{\result}{\mbox{$\mapsto$}}
\newcommand*{\squiggle}{\char126}

\newcommand*{\application}[1]{\mbox{\textsf{#1}}}
\newcommand*{\conj}[1]{{#1}^\ast}         % complex conjugate
\newcommand*{\eqnref}[1]{(\ref{#1})}
\newcommand{\fixme}[1]{\ifhevea\colorbox{\fixmebackgroundcolor}{FIX}\else\fbox{FIX}\fi
                       {#1}%
                       \ifhevea\colorbox{\fixmebackgroundcolor}{ME}\else\fbox{ME}\fi}
\newcommand*{\fourier}[1]{\widehat{#1}}   % fourier transformed variable
\newcommand{\fullref}[1]{\ifhevea \ref{#1}\else \ref{#1}, \pagename~\pageref{#1}\fi}
\newcommand*{\guielement}[1]{\textsc{#1}}
%%\newcommand*{\hangingitem}{\ifhevea\relax\else\hangindent=2em\hangafter=1\fi}
\newcommand{\itempar}[1]{\ifhevea #1\else \parbox[t]{.98\linewidth}{#1\vspace{.667ex}}\fi}
\newcommand{\mainpurpose}[1]{\strong{Main Purpose:}~#1}
\newcommand*{\oldstyleNth}[1]{\mbox{${#1}^{\mathrm{th}}$}}
\newcommand*{\optional}[1]{$[$#1$]$}
\newcommand*{\propername}[1]{\textsc{#1}}
\newcommand{\restrictednote}[1]{\ifhevea\colorbox{\restrictedmaterialbordercolor}{#1}\else\fbox{\textrm{#1}}\fi}
\newcommand{\shifteven}[1]% shift a box down half of \baselineskip
           {\setbox1=\hbox{#1}\raisebox{-.5\rigidbaselineskip}[\ht1][\dp1]{#1}}
\newcommand*{\transpose}[1]{{#1}^{\mathrm{T}}}         % transposition

\newcommand*{\flexipageref}[2]{\ifhevea\ahrefloc{#2}{#1}\else #1, \pagename~\pageref{#2}\fi}
\WithSuffix\newcommand\flexipageref*[2]{\ifhevea\ahrefloc{#2}{#1}\else #1 (\pagename~\pageref{#2})\fi}
\newcommand*{\semilog}[2]{\mbox{${#1}\cdot 10^{#2}$}} % \cdot or \times
\newcommand*{\slfrac}[2]% Knuth, answer to Question 11.6 rewritten with LaTeX macros
{\ifhevea #1/#2\else
  \hspace{.1em}\raisebox{.5ex}{\scriptsize #1}\hspace{-.1em}/%
  \hspace{-.15em}\raisebox{-.25ex}{\scriptsize #2}%
 \fi}


%% Source-of-document-date on the title page
\newcommand*{\datebyunknown}{\footenote{Date determined in an unreliable way}}
\newcommand*{\datebyfs}{\footenote{Date determined via file-system -- potentially unreliable}}
\newcommand*{\datebyscm}{\footnote{Date determined via source-code management system}}


%% Inclusion of graphics and images.  Argument #1 passes additional
%% directives to the sub-macro; argument #2 is the image filename.
\newcommand{\includeimage}[2][]%
{\ifhevea
   \imgsrc{#2.png}%
 \else
   \IfFileExists{#2.pstex}%
                {\input{#2.pstex}}% EPS(La)TeX file transporting PostScript inside a LaTeX envelope
                {\includegraphics[#1]{#2}}% Usually an Encapsulated PostScript file
 \fi}


%% Pacify Hevea
\ifhevea
  \def\WithSuffix{\relax}
  \def\caption[#1]#2{#2}
  \def\endappendix{\relax}
  \def\endexample{\relax}
  \def\endsavenotes{\relax}
  \def\example{\relax}
  \def\floatname{\relax}
  \def\floatstyle{\relax}
  \def\fontseries{\relax}
  \def\hyp{-}
  \def\listof{\relax}
  \def\makesavenoteenv{\relax}
  \def\newfloat{\relax}
  \def\obeylines{\relax}
  \def\overfullrule=#1{\relax}
  \def\savenotes{\relax}
  \def\selectfont{\relax}
  \def\slash{/}
\fi


\ifreferencemanual
  %% Class Refrep (``Reference Manual'') has no twocolumn layout.
  \def\twocolumn[#1]{#1}
\fi


%% Allow for some hyphenation of typewriter fonts.
\newenvironment{makehyphenatable}[1]{#1\hyphenchar\the\font=`\-}{}


%% Taken from Refman style and adapted.
\newcommand{\declaremaxipagerule}{\newif\ifmaxipagerule\maxipageruletrue}% See Knuth, p211

\ifreferencemanual
  \relax
\else
  \declaremaxipagerule

  \newlength{\maxipagewidth}
  \setlength{\maxipagewidth}{\textwidth}
  \addtolength{\maxipagewidth}{1.9\oddsidemargin}

  \newenvironment{maxipage}%
  {\par
    \makebox[-.95\oddsidemargin]{}%
    \begin{minipage}{\maxipagewidth}%
    \medskip
    \ifhevea
      \begin{divstyle}{maxipage}%
    \else
      \ifmaxipagerule \hrule\medskip \fi
      \parskip=.5\baselineskip
    \fi
    \def\marginpar{\typeout{Marginpar not allowed within `maxipage' environment.}}}%
  {\par
    % TeX: \vskip\parskip
    \medskip
    \ifhevea
      \end{divstyle}%
    \else
      \ifmaxipagerule \hrule\medskip \fi
    \fi
    \end{minipage}%
    \par}%
\fi


%% Program listings

\lstdefinestyle{colors}{}

\lstnewenvironment{cxxlisting}%
{\setenvclass{lstlisting}{cxxlisting}
\ifhevea
  \lstset{language=C++,
          commentstyle=\color{orangered}\textsl,
          keywordstyle=\color{violetblue}\textbf,
          stringstyle=\color{orchid},
          style=colors}
\else
  \lstset{language=C++,
          basicstyle={\small\ttfamily},
          commentstyle=\textsl,
          keywordstyle=\textbf}
\fi}%
{}

\lstnewenvironment{adalisting}%
{\setenvclass{lstlisting}{adalisting}
\ifhevea
  \lstset{language=Ada,
          style=colors}
\else
  \lstset{language=Ada,
          basicstyle={\small\ttfamily},
          commentstyle=\textsl,
          keywordstyle=\textbf}
\fi}%
{}


%% Description list with definitions marked with `\code'.
\newenvironment{codelist}%
{\begin{savenotes}\begin{list}{}{\renewcommand{\makelabel}[1]{\code{##1}}}}%
{\end{list}\end{savenotes}}
\makesavenoteenv{codelist}


%% Material that is not essential, but that we do not want to see as
%% footnotes.
\newcommand*{\geeknotemarkup}[1]{\colorbox{\geeknotebackgroundcolor}{\small #1}}

\ifhevea
  \newenvironment{geeknote}%
                 {\begin{divstyle}{geeknote}\begin{quotation}\small}%
                 {\end{quotation}\end{divstyle}}
\else
  \newenvironment{geeknote}%
                 {\begin{savenotes}\fboxsep=0pt%
                  \begin{bgcolor}{\geeknotebackgroundcolor}\rule{\fill}{0pt}%
                  \begin{quotation}%
                  \small\vspace*{\smallskipamount}}%
                 {\vspace*{\smallskipamount}%
                  \end{quotation}%
                  \end{bgcolor}%
                  \end{savenotes}}
  \makesavenoteenv{geeknote}
\fi


%% Literal text as e.g. the contents of a text file.
\newenvironment{literal}%
               {\pagebreak[1]\begin{divstyle}{literal}%
                   \begin{trivlist}{}{\ifhevea\relax\else\topsep=0pt\fi}\item\relax
                     \RaggedRight\frenchspacing\ttfamily}%
               {\end{trivlist}\end{divstyle}}


%% Description list for summarizing options.
\newenvironment{optionsummary}
{\medskip\pagebreak[1]\par\noindent\strong{Summary of influential options}%
\begin{list}{}{\ifhevea\relax\else\topsep=0pt\leftmargin=4em\fi
    \renewcommand{\makelabel}[1]{\option{##1}:\ }}}%
{\end{list}}


%% Markup for material that applies only to a particular application,
%% this is either Enblend or Enfuse, but not both.  The markup
%% e.g. applies to different operating systems, too.
\newenvironment{restrictedmaterial}[1]%
{\begin{savenotes}\begin{divstyle}{restrictedmaterial}\noindent\restrictednote{#1}\begin{quotation}}%
{\end{quotation}%
  \ifhevea\relax\else\RaggedLeft\mbox{$\Box$}\par\fi
  \end{divstyle}\end{savenotes}}
\makesavenoteenv{restrictedmaterial}


%% Simulate terminal output.
\newenvironment{terminal}%
{\begin{divstyle}{terminal}\begin{flushleft}\frenchspacing\ttfamily}%
{\end{flushleft}\end{divstyle}}


%% Overwrite `description' environment, because Hevea's built-in
%% definition is broken.  Keep `description' for reference manual
%% styles, but ditch the colon after description items.
\ifreferencemanual
  \descriptioncolonfalse
\else
  \renewenvironment{description}%
  {\begin{list}{}{\renewcommand{\makelabel}[1]{##1}}}%
  {\end{list}}%
\fi


%% Extra floating objects: exemplars.  We cannot use `example',
%% because this name is taken in Reference-Manual style.
\ifhevea
  \input{exemplar.hva}
  \renewcommand{\exemplarname}{Example}
  \newcommand{\listofexemplars}{\relax}
\else
  \usepackage{trivfloat}
  \trivfloat{exemplar}
  \renewcommand{\exemplarname}{Example}
  \renewcommand{\listexemplarname}{List of Examples}
\fi


%% Define a key-value pair with `\hashinsert{KEY}{VALUE}'.  Reference
%% it like any other label, this is, say `\hashfind{KEY}' to recover
%% VALUE.
\def\hashinsert#1{\expandafter\def\csname valuehash(#1)\endcsname}
\def\hashfind#1{\csname valuehash(#1)\endcsname}

\newcommand*{\indicatesourcevalue}[1]{\mbox{$\langle${#1}\/$\rangle$}}
\newcommand*{\val}[1]{\indicatesourcevalue{\hashfind{#1}}}
\WithSuffix\newcommand\val*[1]{\hashfind{#1}}
