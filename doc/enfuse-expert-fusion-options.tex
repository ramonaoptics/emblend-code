%% This file is part of Enblend.
%% Licence details can be found in the file COPYING.


\newcommand*{\metalowercutoff}{\metavar{LOWER\hyp{}CUT\shyp OFF}}
\newcommand*{\metauppercutoff}{\metavar{UP\shyp PER\hyp{}CUT\shyp OFF}}
\newcommand*{\metalowerprojector}{\metavar{LOWER\hyp{}PRO\shyp JEC\shyp TOR}}
\newcommand*{\metaupperprojector}{\metavar{UP\shyp PER\hyp{}PRO\shyp
    JEC\shyp TOR}}



\subsection[Expert Fusion Options]{\label{sec:expert-fusion-options}%
  \genidx[\rangebeginlocation]{expert fusion options}%
  \genidx{options!fusion for experts}%
  Expert Fusion Options}

Expert fusion options control details of contrast-weight algorithms and they set ultimate
cutoffs for entropy and exposure fusion.

\begin{codelist}
  \label{opt:contrast-edge-scale}%
  \optidx[\defininglocation]{--contrast-edge-scale}%
  \genidx{Laplacian-of-Gaussian@\propername{Laplacian}-of-\propername{Gaussian}}%
  \gensee{LoG@\acronym{LoG}}{\propername{Laplacian}-of-\propername{Gaussian}}%
\item[\itempar{--contrast-edge-scale=\metavar{EDGE-SCALE}\optional{:\feasiblebreak
      \metavar{LCE-SCALE}:\feasiblebreak\metavar{LCE-FACTOR}}}]\itemend
  A non-zero value for \metavar{EDGE-SCALE} switches on the
  \propername{Laplacian}\hyp{}of\hyp{}\propername{Gaussian} (\acronym{LoG}) edge detection
  algorithm.  \metavar{EDGE-SCALE} is the radius of the \propername{Gaussian} used in the search
  for edges.  Default: \val{val:default-edge-scale}~pixels.

  \genidx{contrast!local enhancement}%
  \gensee{local contrast enhancement}{contrast, local enhancement}%
  \gensee{LCE@\acronym{LCE}}{local contrast enhancement}%
  A positive \metavar{LCE-SCALE} turns on local contrast enhancement (\acronym{LCE}) before the
  \acronym{LoG} edge detection.  \metavar{LCE-SCALE} is the radius of the \propername{Gaussian}
  used in the enhancement step, \metavar{LCE-FACTOR} is the weight factor (``strength'').
  \App{} calculates the \metavar{enhanced} values of the \metavar{original} ones with

  \begin{adalisting}
enhanced :=
    (1 + LCE-FACTOR) * original -
    LCE-FACTOR * GaussianSmooth(original,
                                LCE-SCALE).
  \end{adalisting}

  \metavar{LCE-SCALE} defaults to \val{val:default-lce-scale}~pixels and \metavar{LCE-FACTOR}
  defaults to \val{val:default-lce-factor}.  Append \sample{\%} to \metavar{LCE-SCALE} to
  specify the radius as a percentage of \metavar{EDGE-SCALE}.  Append \sample{\%} to
  \metavar{LCE-FACTOR} to specify the weight as a percentage.


  \label{opt:contrast-min-curvature}%
  \optidx[\defininglocation]{--contrast-min-curvature}%
  \genidx{contrast!minimum curvature}%
\item[--contrast-min-curvature=\metavar{CURVATURE}]\itemend
  Define the minimum \metavar{CURVATURE} for the \acronym{LoG} edge detection.  Default:
  \val{val:default-minimum-curvature}.  Append a \sample{\%} to specify the minimum curvature
  relative to maximum pixel value in the source image (for example 255 or~65535).

  A positive value makes \App{} use the local contrast data (controlled with
  option~\flexipageref{\option{--contrast-window-size}}{opt:contrast-window-size}) for
  curvatures less than \metavar{CURVATURE} and \acronym{LoG} data for values above it.

  A negative value truncates all curvatures less than $-\metavar{CURVATURE}$ to zero.  Values
  above \metavar{CURVATURE} are left unchanged.  This effectively suppresses weak edges.


  \label{opt:contrast-window-size}%
  \optidx[\defininglocation]{--contrast-window-size}%
  \genidx{contrast!window size}%
  \gensee{window size!contrast}{contrast, window size}%
\item[--contrast-window-size=\metavar{SIZE}]\itemend
  Set the window \metavar{SIZE} for local contrast analysis.  The window will be a square of
  $\metavar{SIZE} \times \metavar{SIZE}$~pixels.  If given an even \metavar{SIZE}, \App{} will
  automatically use the next odd number.

  For contrast analysis \metavar{SIZE} values larger than 5~pixels might result in a blurry
  composite image.  Values of 3 and~5~pixels have given good results on focus stacks.

  Valid range: $\metavar{SIZE} \geq \val{val:minimum-contrast-window-size}$, default:
  \val{val:default-contrast-window-size}~pixels.

  See also \sectionName~\fullref{sec:fusion-options}, options
  \flexipageref{\option{--contrast-weight}}{opt:contrast-weight} and
  \flexipageref{\option{--hard-mask}}{opt:hard-mask}.


  \label{opt:entropy-cutoff}%
  \optidx[\defininglocation]{--entropy-cutoff}%
  \genidx{entropy!cutoff}%
  \gensee{cutoff entropy}{entropy, cutoff}%
\item[--entropy-cutoff=\metalowercutoff\optional{:\metauppercutoff{}}]\itemend
  Define a cutoff function~$Y'$ for the normalized luminance~$Y$ by \metalowercutoff{} and
  \metauppercutoff, which gets applied (only) before the local-entropy calculation.
  \metalowercutoff{} is the value below which pixels are mapped to pure black when calculating
  the local entropy of the pixel's surroundings.  Optionally also define the \metauppercutoff{}
  value above which pixels are mapped to pure white.

  \begin{equation}\label{equ:luminance-entropy-cutoff}
  Y' = \left\{
  \begin{array}{ll}
    0 & \quad \mbox{for }Y \le \metalowercutoff, \\
    1 & \quad \mbox{for }Y \ge \metauppercutoff\mbox{ and} \\
    Y & \quad \mbox{otherwise.}
  \end{array}
  \right.
  \end{equation}

  Also see \sectionName~\fullref{sec:local-entropy-weighting} for an explanation of local
  entropy.  \figureName~\fullref{fig:entropy-cutoff} shows an example for the luminance mapping.

  Note that the entropy cutoff does not apply directly to the local-entropy~$H$ of a pixel or
  its weight~$w_H$, but the luminance image that get fed into the local-entropy weight
  calculation.  However, assigning \emph{constant} values to extreme shadows or highlights
  in general decreases their local entropy, thereby reducing the pixels' weights.

  For color images \metalowercutoff{} and \metauppercutoff{} are applied separately and
  independently to each channel.

  Append a \sample{\%}-sign to specify the cutoff relative to maximum pixel value in the source
  image (for example 255 or~65535).  Negative \metauppercutoff{} values indicate the maximum
  minus the absolute \metauppercutoff{} value; the same holds for negative percentages.

  Defaults: \val{val:default-entropy-lower-cutoff} for \metalowercutoff{} and
  \val{val:default-entropy-upper-cutoff} for \metauppercutoff, that is, all pixels' values are
  taken into account.

  \begin{figure}
    \begin{maxipage}
      \centering
      \includeimage{entropy-cutoff}
    \end{maxipage}

    \caption[Entropy cutoff function]{\label{fig:entropy-cutoff}%
      Modified lightness~$Y'$, \equationabbr~\ref{equ:luminance-entropy-cutoff}, for
      $\metalowercutoff = 5\%$ and $\metauppercutoff = 90\%$, which are rather
      extreme values.}
  \end{figure}

  \begin{geeknote}
    Note that a high \metalowercutoff{} value lightens the resulting image, as dark and
    presumably noisy pixels are averaged with \emph{equal} weights.  With $\metalowercutoff =
    0$, the default, on the other hand, ``noise'' might be interpreted as high entropy and the
    noisy pixels get a high weight, which in turn renders the resulting image darker.
    Analogously, a low \metauppercutoff{} darkens the output image.
  \end{geeknote}

  \label{opt:entropy-window-size}%
  \optidx[\defininglocation]{--entropy-window-size}%
  \genidx{entropy!window size}%
  \gensee{window size!entropy}{entropy, window size}%
\item[--entropy-window-size=\metavar{SIZE}]\itemend
  Window \metavar{SIZE} for local entropy analysis.  The window will be a square of
  $\metavar{SIZE} \times \metavar{SIZE}$~pixels.

  In the entropy calculation \metavar{SIZE} values of 3 to~7 yield an acceptable compromise of
  the locality of the information and the significance of the local entropy value itself.

  Valid range: $\metavar{SIZE} \geq \val{val:minimum-entropy-window-size}$, default:
  \val{val:default-entropy-window-size}~pixels.

  If given an even \metavar{SIZE} \App{} will automatically use the next-larger odd number.


  \label{opt:exposure-cutoff}%
  \optidx[\defininglocation]{--exposure-cutoff}%
  \genidx{exposure!cutoff}%
  \gensee{cutoff!exposure}{exposure, cutoff}%
\item[\itempar{--exposure-cutoff=\metalowercutoff\optional{:\feasiblebreak
      \metavar{UPPER\hyp{}CUTOFF}\optional{:\feasiblebreak
        \metalowerprojector:\feasiblebreak
        \metaupperprojector}}}]\itemend
  Define an exposure-cutoff function by the luminances \metalowercutoff{} and \metauppercutoff.
  Pixels below the lower or above the upper cutoff get a weight of exactly zero irrespective of
  the active exposure-weight function.

  For color images the values of \metalowercutoff{} and \metauppercutoff{} refer to the
  gray-scale projection as selected with \metalowerprojector{} and \metaupperprojector.  This is
  similar to option~\option{--gray\hyp{}pro\shyp jec\shyp tor}.

  Append a \sample{\%}-sign to specify the cutoff relative to maximum pixel value in the source
  image (for example 255 or~65535).  Negative \metauppercutoff{} values indicate the maximum
  minus the absolute \metauppercutoff{} value; the same holds for negative percentages.

  \begin{geeknote}
    The impact of this option is similar, but not identical to transforming \emph{all} input
    images with \uref{\imagemagickorg}{ImageMagick's}\appidx{ImageMagick}
    \command{convert}\prgidx{convert} (see \appendixName~\fullref{sec:helpful-programs}) prior
    to fusing with the commands demonstrated in
    \exampleName~\ref{ex:imagemagick-convert-cutoff}.

    \begin{exemplar}
      \begin{terminal}
        \$ convert IMAGE \bslash \\
        ~~~~~~~~~~~\bslash( +clone -threshold LOWER-CUTOFF \bslash) \bslash \\
        ~~~~~~~~~~~-compose copy\_opacity -composite \bslash \\
        ~~~~~~~~~~~MASKED-IMAGE
      \end{terminal}

      \smallskip

      \begin{terminal}
        \$ convert IMAGE \bslash \\
        ~~~~~~~~~~~\bslash( \bslash \\
        ~~~~~~~~~~~~~~~\bslash( IMAGE -threshold LOWER-CUTOFF \bslash) \bslash \\
        ~~~~~~~~~~~~~~~\bslash( IMAGE -threshold UPPER-CUTOFF -negate \bslash) \bslash \\
        ~~~~~~~~~~~~~~~-compose multiply -composite \bslash \\
        ~~~~~~~~~~~\bslash) \bslash \\
        ~~~~~~~~~~~-compose copy\_opacity -composite \bslash \\
        ~~~~~~~~~~~MASKED-IMAGE
      \end{terminal}

      \caption[\application{ImageMagick} for exposure cutoff]%
              {\label{ex:imagemagick-convert-cutoff}%
                Using \application{ImageMagick} for exposure cutoff operations.  The first
                example only applies a lower cutoff, whereas the second one applies both a
                lower and an upper cutoff to the images.}
    \end{exemplar}

    Transforming some or all input images as shown in the example gives the user more
    flexibility because the thresholds can be chosen for each image individually.
  \end{geeknote}

  The option allows to specify projection operators as in
  option~\option{--gray\hyp{}pro\shyp jec\shyp tor} for the
  \metalowercutoff{} and \metauppercutoff{} thresholds.

  This option can be helpful if the user wants to exclude underexposed or overexposed pixels
  from the fusing process in \emph{all} of the input images.  The values of
  \metalowercutoff{} and \metauppercutoff{} as well as the gray-scale projector
  determine which pixels are considered ``underexposed'' or ``overexposed''.  As any change of
  the exposure-weight curve this option changes the brightness of the resulting image:
  increasing \metalowercutoff{} lightens the final image and lowering \metauppercutoff{}
  darkens it.

  Defaults: \val{val:default-exposure-lower-cutoff} for \metalowercutoff{} and
  \val{val:default-exposure-upper-cutoff} for \metauppercutoff, that is, all pixels' values are
  weighted according to the ``uncut'' exposure-weight curve.

  \figureName~\ref{fig:exposure-cutoff} shows an example.

  The gray-scale projectors \metalowerprojector{} and \metaupperprojector{} default to
  \sample{\val{val:default-exposure-lower-cutoff-projector}} and
  \sample{\val{val:default-exposure-upper-cutoff-projector}}, which are usually the best choices
  for effective cutoff operations on the respective ends.

  \begin{figure}
    \begin{maxipage}
      \centering
      \includeimage{exposure-cutoff}
    \end{maxipage}

    \caption[Exposure cutoff function]{\label{fig:exposure-cutoff}%
      Exposure weight~$w_Y$ after an exposure-cutoff of $\metalowercutoff = 5\%$ and
      $\metauppercutoff = 97\%$ was applied to a \propername{Gaussian} with the
      $\metavar{optimum} = 0.5$ and $\metavar{width} = 0.2$.}
  \end{figure}

  Note that the application of the respective cutoffs is completely independent of the actual
  shape of the exposure weight function.

  If a set of images stubbornly refuses to ``react'' to this option, look at their histograms to
  verify the cutoff actually falls into populated ranges of the histograms.  In the absence of
  an image manipulation program like \uref{\gimporg}{\application{The Gimp}}\appidx{Gimp},
  \uref{\imagemagickorg}{ImageMagick's}\appidx{ImageMagick} can be used to generate
  \uref{\imagemagickorgusagefileshistogram}{histograms}, like, for example,

  \begin{terminal}
    \$ convert -define histogram:unique-colors=false \bslash \\
    ~~~~~~~~~~~IMAGE histogram:- | \bslash \\
    ~~~~~~~display
  \end{terminal}

  The syntax of this option is flexible enough to combine ease of use and precision, as
  \tableName~\ref{tab:flexible-exposure-cutoff} demonstrates.

  \begin{table}
    \begin{maxipage}
      \begin{tabular}{p{.3\linewidth}lp{.3\linewidth}}
        \hline
        \multicolumn{1}{c|}{Task} &
        \multicolumn{1}{c|}{Cutoff Setting} &
        \multicolumn{1}{c}{Effect} \\
        \hline\extraheadingsep
        Suppress some noise. & \option{--exposure-cutoff=5\%} & The percentage makes the cutoff
        specification channel-width agnostic. \\

        Shave off pure white pixels. & \option{--exposure-cutoff=0:-1} & This cutoff
        specification only works for integral pixels, but it will set the weight of the very
        brightest pixels to zero. \\

        Shave off bright white pixels. & \option{--exposure-cutoff=0:-1\%} & Here we exclude the
        brightest 1\% of pixels form the exposure fusion no matter whether the image is encoded
        with integers or floating-point numbers. \\

        Suppress some noise and shave off pure white pixels. & \option{--exposure-cutoff=5\%:-1}
        & Combine the effects of lower and upper cutoff, while mixing relative and absolute
        specifications.
      \end{tabular}
    \end{maxipage}

    \caption[Flexible exposure cutoff]{\label{tab:flexible-exposure-cutoff}%
      Some possible exposure-cutoff settings and their effects on the exposure weights.}
  \end{table}


  \label{opt:exposure-weight-function}%
  \optidx[\defininglocation]{--exposure-weight-function}%
  \genidx{weight!function!exposure}%
  \gensee{exposure weight!function}{weight, function, exposure}%
\item[\itempar{--exposure-weight-function=\metavar{WEIGHT-FUNCTION}~\textrm{(\oldstylefirst~form)}
    \\ --exposure-weight-function=\metavar{SHARED-OBJECT}:\feasiblebreak
    \metavar{SYMBOL}\optional{:\feasiblebreak
      \metavar{ARGUMENT}\optional{:\dots}}~\textrm{(\oldstylesecond~form)}}]\itemend
  \oldstylefirst~form: override the default exposure weight function
  (\code{\val{val:exposure-weight-function}}) and instead use one of the weight-functions in
  \tableName~\fullref{tab:exposure-weight-functions}.

  \restrictednote{Versions with dynamic-linking support only.}  See
  \sectionName~\fullref{sec:finding-out-details} on how to check for this feature.

  \genidx{shared object}%
  \gensee{object!shared}{shared object}%
  \noindent \oldstylesecond~form: dynamically load \metavar{SHARED-OBJECT} and use
  \metavar{SYMBOL} as user-defined exposure weight function.  Optionally pass the user-defined
  function \metavar{ARGUMENT}s.

  \genidx{dynamic library}%
  \gensee{library!dynamic}{dynamic library}%
  \begin{geeknote}
    Depending on the operating system environment, a ``shared object'' is sometimes also called
    a ``dynamic library''.
  \end{geeknote}

  \genidx{exposure weight!linear transform}%

  In \tableName~\ref{tab:exposure-weight-functions} the variable~$w_{\mathrm{exp}}$ denotes the
  exposure weight and $z$ represents the normalized luminance~$Y$ linearly transformed by the
  exposure optimum~$Y_{\mathrm{opt}}$ (option~\option{--exposure-optimum}) and \metavar{width}
  (option~\option{--exposure-width}) according to the linear transform
  \begin{equation}\label{equ:linear-luminance-transform}
  z = \frac{Y - Y_{\mathrm{opt}}}{\mathit{width}}.
  \end{equation}

 \genidx{exposure weight function!full width half maximum}%
 \genidx{exposure weight function!\acronym{FWHM}}%
 \genidx{full width half maximum}%
 \gensee{FWHM@\acronym{FWHM}}{full width half maximum}%
 Internally \App{} uses a rescaled \metavar{width} that gives all weight functions the same full
 width at half of the maximum (\acronym{FWHM}), also see \figureName~\ref{fig:exposure-weights}.
 This means for the user that changing the exposure function neither changes the optimum
 exposure nor the width.

  \begin{table}
    \begin{codelist}
      \genidx{exposure weight function!gauss}%
      \genidx{exposure weight function!gaussian}%
    \item[\itempar{gauss \\ gaussian}]\itemend
      The original exposure weight function of \App{} and the only one up to version~4.1.
      \begin{equation}\label{equ:weight:gauss}
      w_{\mathrm{exp}} = \exp\left({-z^2 / 2}\right)
      \end{equation}

      \genidx{exposure weight function!lorentz}%
      \genidx{exposure weight function!lorentzian}%
    \item[\itempar{lorentz \\ lorentzian}]\itemend
      Lorentz curve.
      \begin{equation}\label{equ:weight:lorentz}
      w_{\mathrm{exp}} = \frac{1}{1 + z^2 / 2}
      \end{equation}

      \genidx{exposure weight function!halfsine}%
      \genidx{exposure weight function!half-sine}%
    \item[\itempar{halfsine \\ half-sine}]\itemend
      Upper half-wave of a sine; exactly zero outside.
      \begin{equation}\label{equ:weight:halfsine}
      w_{\mathrm{exp}} =
      \left\{\begin{array}{cl}
      \cos(z) & \mbox{if } |z| \leq \pi/2 \\
      0       & \mbox{otherwise.}
      \end{array}\right.
      \end{equation}

      \genidx{exposure weight function!fullsine}%
      \genidx{exposure weight function!full-sine}%
    \item[\itempar{fullsine \\ full-sine}]\itemend
      Full sine-wave shifted upwards by one to give all positive weights; exactly zero outside.
      \begin{equation}\label{equ:weight:fullsine}
      w_{\mathrm{exp}} =
      \left\{\begin{array}{cl}
      (1 + \cos(z)) / 2 & \mbox{if } |z| \leq \pi \\
      0                 & \mbox{otherwise.}
      \end{array} \right.
      \end{equation}

      \genidx{exposure weight function!bisquare}%
      \genidx{exposure weight function!bi-square}%
    \item[\itempar{bisquare \\ bi-square}]\itemend
      Quartic function.
      \begin{equation}\label{equ:weight:bisquare}
      w_{\mathrm{exp}} =
      \left\{
      \begin{array}{cl}
        1 - z^4 & \mbox{if } |z| \leq 1 \\
        0       & \mbox{otherwise.}
      \end{array}
      \right.
      \end{equation}
    \end{codelist}

    \caption[Exposure weight functions]{\label{tab:exposure-weight-functions}%
      \genidx[\summarylocation]{exposure weight function}%
      Predefined exposure weight functions.  For a graphical comparison see
      \figureName~\ref{fig:exposure-weights}.}
  \end{table}

  For a detailed explanation of all the weight functions
  \sectionName~\fullref{sec:exposure-weighting}.

  If this option is given more than once, the last instance wins.


  \label{opt:gray-projector}%
  \optidx[\defininglocation]{--gray-projector}%
  \genidx{gray projector}%
  \gensee{projector to grayscale}{gray projector}%
\item[--gray-projector=\metavar{PROJECTOR}]\itemend
  Use gray projector~\metavar{PROJECTOR} for conversion of \acronym{RGB} images to grayscale:
  \[
  (R, G, B) \rightarrow Y.
  \]

  In this version of \App{}, the option is effective for exposure weighting and local contrast
  weighting and \metavar{PROJECTOR} defaults to \sample{\val{val:default-grayscale-accessor}}.

  Valid values for \metavar{PROJECTOR} are:

  \begin{codelist}
    \genidx{gray projector!anti-value@\code{anti-value}}%
    \gensee{anti-value gray projector@\code{anti-value} gray projector}%
           {gray projector, \code{anti-value}}%
  \item[anti-value]\itemend
    Do the opposite of the \sample{value} projector: take the minimum of all color channels.
    \[
    Y = \min(R, G, B)
    \]
    This projector can be useful when exposure weighing while employing a lower cutoff (see
    option~\option{--exposure-cutoff}) to reduce the noise in the fused image.

    \genidx{gray projector!average@\code{average}}%
    \gensee{average gray projector@\code{average} gray projector}%
           {gray projector, \code{average}}%
  \item[average]\itemend
    Average red, green, and blue channel with equal weights.  This is the default, and it often
    is a good projector for $\mbox{gamma} = 1$ data.
    \[
    Y = \frac{R + G + B}{3}
    \]

    \genidx{gray projector!channel-mixer@\code{channel-mixer}}%
    \gensee{channel-mixer gray projector@\code{channel-mixer} gray projector}%
           {gray projector, \code{channel-mixer}}%
  \item[\itempar{channel-mixer:\metavar{RED-WEIGHT}:\feasiblebreak
    \metavar{GREEN-WEIGHT}:\feasiblebreak\metavar{BLUE-WEIGHT}}]\itemend
    Weight the channels as given.
    \[
    \begin{array}{r@{\hspace{1ex}}c@{\hspace{1ex}}l@{\hspace{1ex}}c@{\hspace{1ex}}cc}
      Y & = & \metavar{RED-WEIGHT}   & \times & R & + \\
        &   & \metavar{GREEN-WEIGHT} & \times & G & + \\
        &   & \metavar{BLUE-WEIGHT}  & \times & B &
    \end{array}
    \]

    The weights are automatically normalized to one, so

    \begin{literal}
      --gray-projector=channel-mixer:0.25:0.5:0.25 \\
      --gray-projector=channel-mixer:1:2:1 \\
      --gray-projector=channel-mixer:25:50:25
    \end{literal}

    all define the same mixer configuration.

    The three weights \metavar{RED-WEIGHT}, \metavar{GREEN\hyp{}WEIGHT}, and
    \metavar{BLUE\hyp{}WEIGHT} define the relative weight of the respective color channel.  The
    sum of all weights is normalized to one.

    \genidx{gray projector!l-star@\code{l-star}}%
    \gensee{l-star gray projector@\code{l-star} gray projector}%
           {gray projector, \code{l-star}}%
    \genidx{RGB-L*a*b* conversion@\acronym{RGB}-\acronym{L*a*b*} conversion}%
    \gensee{conversion!RGB-L*a*b*@\acronym{RGB}-\acronym{L*a*b*}}%
           {\acronym{RGB}-\acronym{L*a*b*} conversion}%
  \item[l-star]\itemend
    Use the L-channel of the L*a*b*-conversion of the image as its grayscale representation.
    This is a useful projector for $\mbox{gamma} = 1$ data.  It reveals minute contrast
    variations even in the shadows and the highlights.  This projector is computationally
    expensive.  Compare with \sample{pl-star}, which is intended for gamma-corrected images.

    See \uref{\wikipedialabcolorspace}{Wikipedia} for a detailed description of the
    \acronym{Lab}~color space.

    \genidx{gray projector!lightness@\code{lightness}}%
    \gensee{lightness gray projector@\code{lightness} gray projector}%
           {gray projector, \code{lightness}}%
  \item[lightness]\itemend
    Compute the lightness of each \acronym{RGB} pixel as in an
    Hue\hyp{}Saturation\hyp{}Lightness (\acronym{HSL}) conversion of the image.
    \[
    Y = \frac{\max(R, G, B) + \min(R, G, B)}{2}
    \]

    \genidx{gray projector!luminance@\code{luminance}}%
    \gensee{luminance gray projector@\code{luminance} gray projector}%
           {gray projector, \code{luminance}}%
  \item[luminance]\itemend
    Use the weighted average of the \acronym{RGB} pixel's channels as defined by \acronym{CIE}
    (``Commission Internationale de l'\'Eclairage'') and the \acronym{JPEG} standard.
    \[
    Y = 0.30 \times R + 0.59 \times G + 0.11 \times B
    \]

    \genidx{gray projector!pl-star@\code{pl-star}}%
    \gensee{pl-star gray projector@\code{pl-star} gray projector}%
           {gray projector, \code{pl-star}}%
    \genidx{RGB'-L*a*b* conversion@\acronym{RGB'}-\acronym{L*a*b*} conversion}%
  \item[pl-star]\itemend
    Use the L-channel of the L*a*b*-conversion of the image as its grayscale representation.
    This is a useful projector for gamma-corrected data.  It reveals minute contrast variations
    even in the shadows and the highlights.  This projector is computationally expensive.
    Compare with \sample{l-star}, which is intended for $\mbox{gamma} = 1$ images.

    See \uref{\wikipedialabcolorspace}{Wikipedia} for a detailed description of the
    \acronym{Lab}~color space.

    \genidx{gray projector!value@\code{value}}%
    \gensee{value gray projector@\code{value} gray projector}{gray projector, \code{value}}%
  \item[value]\itemend
    Take the Value-channel of the Hue-Saturation-Value (\acronym{HSV}) conversion of the image.
    \[
    Y = \max(R, G, B)
    \]
  \end{codelist}
\end{codelist}

\genidx[\rangeendlocation]{expert fusion options}


%%% Local Variables:
%%% fill-column: 96
%%% End:
