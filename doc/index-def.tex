%% -*- latex -*-


%% This file is part of Enblend.
%% Licence details can be found in the file COPYING.


\usepackage{index}


\makeindex

\newindex{general}{gdx}{gnd}{General Index}
\newindex{option}{odx}{ond}{Option Index}                  % command-line options
\newindex{program}{pdx}{pnd}{Program/\hspace{-.167em}Application Index}
\newindex{syncomm}{adx}{and}{Syntactic Comment Index}      % syntactic comments in response files


%%
%% Level 1 Index Generation
%%


\newcommand*{\normallocation}{normal location}
\newcommand*{\defininglocation}{defining location}
\newcommand*{\summarylocation}{summary location}
\newcommand*{\rangebeginlocation}{range-begin location}
\newcommand*{\rangeendlocation}{range-end location}


%% \makefancyindex
%%     #1: name of index (program, option, general, ...)
%%     #2: location macro (\normallocation, \defininglocation, ...)
%%     #3: (possibly decorated) index item
\newcommand*{\makefancyindex}[3]{%
  \ifx\normallocation#2%
    \index[#1]{#3}%
  \else
    \ifx\defininglocation#2%
      \index[#1]{#3|underline}%
    \else
      \ifx\summarylocation#2%
        \index[#1]{#3|textsl}%
      \else
        \ifx\rangebeginlocation#2%
          % Downgrade range-begin to a normal index entry for Hevea/HTML
          \ifhevea \index[#1]{#3}\else \index[#1]{#3|(}\fi
        \else
          \ifx\rangeendlocation#2%
            % Range-ends make little sense for Hevea/HTML
            \ifhevea \relax \else \index[#1]{#3|)}\fi
          \else
            \index[#1]{???: #3}%
          \fi
        \fi
      \fi
    \fi
  \fi
}

\newcommand*{\appidx}[2][\normallocation]% program/application name index: application name
            {\makefancyindex{program}{#1}{\application{#2}}}
\newcommand*{\genidx}[2][\normallocation]{\makefancyindex{general}{#1}{#2}}
\newcommand*{\optidx}[2][\normallocation]% command-line option index
            {\makefancyindex{option}{#1}{\option{#2}}}
\newcommand*{\prgidx}[2][\normallocation]% program/application name index: command name
            {\makefancyindex{program}{#1}{\code{#2}}}
\newcommand*{\shoptidx}[2]{\index[option]{\option{#1} (long: \option{#2})}}
\newcommand*{\synidx}[1]{\index[syncomm]{\code{#1}}}% syntactic comment index


%%
%% Level 2 Index Generation  --  Cross-References
%%


%% \seexref
%%     #1: final index term
%%     #2: page number, which we want to drop
\newcommand*{\seexref}[2]{\textit{see\/} #1}


%% \seealsoxref
%%     #1: final index term
%%     #2: page number
\newcommand*{\seealsoxref}[2]{\newline\textit{see also\/} #1, #2}


%% \makefancyreference
%%     #1: name of index (program, option, general, ...)
%%     #2: one-argument cross-reference macro, e.g. \seexref or \seealsoxref
%%     #3: indirecting index item
%%     #4: final index term
\newcommand*{\makefancyreference}[4]{\index[#1]{#3|#2{#4}}}


\newcommand*{\gensee}[2]{\makefancyreference{general}{seexref}{#1}{#2}}
\newcommand*{\gensea}[2]{\makefancyreference{general}{seealsoxref}{#1}{#2}}


%--\proofmodetrue
