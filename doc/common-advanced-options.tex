%% This file is part of Enblend.
%% Licence details can be found in the file COPYING.


\subsection[Advanced Options\commonpart]{Advanced Options\commonpart
  \label{sec:advanced-options}
  \genidx[\rangebeginlocation]{advanced options}
  \genidx{options!advanced}}

Advanced options control e.g.\ the channel depth, color model, and the
cropping of the output image.

\begin{codelist}
  \label{opt:blend-colorspace}%
  \optidx[\defininglocation]{--blend-colorspace}%
  \genidx{colorspace!blend}%
  \gensee{blend colorspace}{colorspace, blend}%
  \genidx{color appearance model}%
\item[--blend-colorspace=\metavar{COLORSPACE}]\itemend Force blending
  in selected \metavar{COLORSPACE}.  Given well matched images this
  option should not change the output image much.  However, if \App{}
  must blend vastly different colors (as e.g.\ anti-colors) the
  resulting image heavily depends on the \metavar{COLORSPACE}.

  Usually, \App{} chooses defaults depending on the input images:

  \begin{itemize}
  \item
    For grayscale or color input images \emph{with}
    \acronym{ICC}~profiles\genidx{profile!\acronym{ICC}}%
    \gensee{ICC@\acronym{ICC} profile}{profile, \acronym{ICC}} the
    default is to use
    \acronym{CIELUV}~colorspace.\genidx{colorspace!\acronym{CIELUV}}
  \item
    Images \emph{without} color profiles and floating-point images are
    blended in the trivial luminance interval (grayscale) or
    \acronym{RGB}-color cube\genidx{color cube!\acronym{RGB}}%
    \gensee{RGB@\acronym{RGB} color cube}{color cube, \acronym{RGB}}
    by default.
  \end{itemize}

  On the order of fast to slow computation, \App{} supports the
  following blend colorspaces.
  \begin{description}
  \item[\itempar{\code{identity} \\ \code{id} \\ \code{unit}}]
    \itemend Compute blended colors in a na\"ive way sidestepping any
    dedicated colorspace.
    \begin{itemize}
    \item
      Use trivial, 1-dimensional luminance interval\genidx{luminance
        interval!trivial} (see
      Eq.~\fullref{equ:trivial-luminance-blend}) for grayscale images
      and for
    \item
      color images utilize 3-dimensional
      \acronym{RGB}-cube\genidx{color cube!\acronym{RGB}} (see
      Eq.~\fullref{equ:trivial-rgb-blend}) spanned by the input
      \acronym{ICC}~profile or
      \acronym{sRGB}\genidx{sRGB@\acronym{sRGB}} if no profiles are
      present.  In the latter case, consider passing
      option~\flexipageref{\option{--fallback-profile}}{opt:fallback-profile}
      to force a different profile than \acronym{sRGB} upon all input
      images.
    \end{itemize}

  \item[\itempar{\code{lab} \\ \code{cielab} \\ \code{lstar}
      \\ \code{l-star}}] \itemend Blend pixels in the
    \acronym{CIEL*a*b*}\genidx{colorspace!\acronym{CIEL*a*b*}}%
    \gensee{CIEL*a*b*@\acronym{CIEL*a*b*} colorspace}{colorspace, \acronym{CIEL*a*b*}}
    colorspace.

  \item[\itempar{\code{luv} \\ \code{cieluv}}] \itemend Blend pixels
    in the \acronym{CIEL*u*v*}\genidx{colorspace!\acronym{CIEL*u*v*}}%
    \gensee{CIEL*u*v*@\acronym{CIEL*u*v*} colorspace}{colorspace, \acronym{CIEL*u*v*}}
    colorspace.

  \item[\itempar{\code{ciecam} \\ \code{ciecam02} \\ \code{jch}}]
    \itemend Blend pixels in the
    \acronym{CIECAM02}\genidx{colorspace!\acronym{CIECAM02}}%
    \gensee{CIECAM02@\acronym{CIECAM02} colorspace}{colorspace, \acronym{CIECAM02}}
    colorspace.
  \end{description}

  \ifenblend
    \begin{restrictedmaterial}{\application{Enblend} only.}
      Please keep in mind that by using different blend colorspaces,
      blending may not only change the colors of the output image, but
      \application{Enblend} may choose different seam line routes as
      some seam-line optimizers\genidx{optimizer!seam-line} are guided
      by image differences, which are different when viewed in
      different colorspaces.
    \end{restrictedmaterial}
  \fi


  \label{opt-ciecam}%
  \optidx{--ciecam}%
  \shoptidx{-c}{--ciecam}%
\item[\itempar{-c \\ --ciecam}]\itemend Deprecated.  Use
  \sample{--blend-colorspace=ciecam} instead.  To emulate the negated
  option~\option{--no-ciecam}\optidx{--no-ciecam} use
  \begin{literal}
    --blend-colorspace=identity
  \end{literal}

  \label{opt:depth}%
  \optidx[\defininglocation]{--depth}%
  \shoptidx{-d}{--depth}%
\item[\itempar{-d \metavar{DEPTH} \\ --depth=\metavar{DEPTH}}]\itemend
  Force the number of bits per channel\genidx{bits per channel} and
  the numeric format of the output image, this is, the
  \metavar{DEPTH}.  The number of bits per channel is also known as
  ``channel width''\gensee{channel!width}{channel, depth} or ``channel
  depth''.\genidx{channel!depth}

  \App{} always uses a smart way to change the channel depth to assure
  highest image quality at the expense of memory, whether
  requantization\genidx{requantization} is implicit because of the
  output format or explicit through option~\option{--depth}.

  \begin{itemize}
  \item
    If the output-channel depth is larger than the input-channel depth
    of the input images, the input images' channels are widened to the
    output channel depth immediately after loading, that is, as soon
    as possible.  \App{} then performs all blending operations at the
    output-channel depth, thereby preserving minute color details
    which can appear in the blending areas.

  \item
    If the output-channel depth is smaller than the input-channel
    depth of the input images, the output image's channels are
    narrowed only right before it is written to the output
    \metavar{FILE}, that is, as late as possible.  Thus the data
    benefits from the wider input channels for the longest time.
  \end{itemize}

  All \metavar{DEPTH} specifications are valid in lowercase as well as
  uppercase letters.  For integer format, use

  \begin{description}
  \item[\itempar{\code{8} \\ \code{uint8}}]\itemend
    Unsigned 8~bit; range: $0\dots255$

  \item[\code{int16}]\itemend
    Signed 16~bit; range: $-32768\dots32767$

  \item[\itempar{\code{16} \\ \code{uint16}}]\itemend
    Unsigned 16~bit; range: $0\dots65535$

  \item[\code{int32}]\itemend
    Signed 32~bit; range: $-2147483648\dots2147483647$

  \item[\itempar{\code{32} \\ \code{uint32}}]\itemend
    Unsigned 32~bit; range: $0\dots4294967295$
  \end{description}

  %% Minimum positive normalized value: 2^(2 - 2^k)
  %% Epsilon: 2^(1 - n)
  %% Maximum finite value: (1 - 2^(-n)) * 2^(2^k)
  For floating-point format, use

  \begin{description}
  \item[\itempar{\code{r32} \\ \code{real32} \\ \code{float}}]\itemend
    %% IEEE single: 32 bits, n = 24, k = 32 - n - 1 = 7
    \acronym{IEEE754}\genidx{IEEE754@\acronym{IEEE754}!single precision float}
    single\gensee{single precision float (\acronym{IEEE754})}{\acronym{IEEE754}, single precision float}
    precision floating-point, 32~bit wide, 24~bit
    significant;

    \begin{itemize}
    \item
      Minimum normalized value: \semilog{1.2}{-38}
    \item
      Epsilon: \semilog{1.2}{-7}
    \item
      Maximum finite value: \semilog{3.4}{38}
    \end{itemize}

  \item[\itempar{\code{r64} \\ \code{real64} \\ \code{double}}]\itemend
    %% IEEE double: 64 bits, n = 53, k = 64 - n - 1 = 10
    \acronym{IEEE754}\genidx{IEEE754@\acronym{IEEE754}!double precision float}
    double\gensee{double precision float (\acronym{IEEE754})}{\acronym{IEEE754}, double precision float}
    precision floating-point, 64~bit wide, 53~bit
    significant;

    \begin{itemize}
    \item
      Minimum normalized value: \semilog{2.2}{-308}
    \item
      Epsilon: \semilog{2.2}{-16}
    \item
      Maximum finite value: \semilog{1.8}{308}
    \end{itemize}
  \end{description}

  If the requested \metavar{DEPTH} is not supported by the output file
  format, \App{} warns and chooses the \metavar{DEPTH} that matches
  best.

  \restrictednote{Versions with \acronym{OpenEXR} read\slash write
    support only.\footnotemark}\footnotetext{Check with \code{\app{}
      --show-image-formats} and look for \sample{EXR}.}

  \noindent The \acronym{OpenEXR} data
  format\genidx{OpenEXR@\acronym{OpenEXR}!data format} is treated as
  \acronym{IEEE754}~float internally.  Externally, on disk,
  \acronym{OpenEXR} data is represented by ``half'' precision
  floating-point numbers.

  %% ILM half: 16 bits, n = 10, k = 16 - n - 1 = 5
  \uref{\openexrcomfeatures}{\acronym{OpenEXR}}%
  \genidx{OpenEXR@\acronym{OpenEXR}!half precision float}
  half\gensee{half precision float (\acronym{OpenEXR})}{\acronym{OpenEXR}, half precision float}
  precision floating-point, 16~bit wide, 10~bit significant;

  \begin{itemize}
  \item
    Minimum normalized value: \semilog{9.3}{-10}
  \item
    Epsilon: \semilog{2.0}{-3}
  \item
    Maximum finite value: \semilog{4.3}{9}
  \end{itemize}


  \label{opt:f}%
  \optidx[\defininglocation]{-f}%
\item[-f \metavar{WIDTH}x\metavar{HEIGHT}%
  \optional{+x\metavar{XOFFSET}+y\metavar{YOFFSET}}]\itemend Ensure
  that the minimum ``canvas''\genidx{size!canvas}%
  \gensee{canvas size}{size, canvas}
  size of the output\genidx{output image!set size} image is at least
  \metavar{WIDTH}\classictimes\metavar{HEIGHT}.  Optionally specify
  the \metavar{XOFFSET} and \metavar{YOFFSET} of the canvas, too.

  This option only is useful when the input images are cropped
  \acronym{TIFF} files, such as those produced by
  \command{nona}\prgidx{nona \textrm{(Hugin)}}\footnote{The
    stitcher \command{nona} is part of
    \application{Hugin\appidx{hugin}}.}.

  Note that option~\option{-f} neither rescales the output image, nor
  shrinks the canvas size below the minimum size occupied by the union
  of all input images.


  \label{opt:g}%
  \optidx[\defininglocation]{-g}%
\item[-g] Save alpha channel as ``associated''.\genidx{alpha
  channel!associated}\gensee{associated alpha channel}{alpha channel, associated}
  See the \uref{\awaresystemsbeextrasamples}{\acronym{TIFF}
    documentation} for an explanation.

  \application{The Gimp}\appidx{Gimp} before version~2.0 and
  \application{CinePaint}\appidx{Cinepaint} (see
  Appendix~\fullref{sec:helpful-programs}) exhibit unusual behavior
  when loading images with unassociated\gensee{unassociated alpha
    channel}{alpha channel, associated} alpha channels.  Use
  option~\option{-g} to work around this problem.  With this flag
  \App{} will create the output image with the ``associated alpha
  tag'' set, even though the image is really unassociated alpha.


  \label{opt:wrap}%
  \optidx[\defininglocation]{--wrap}%
  \shoptidx{-w}{--wrap}%
\item[\itempar{-w \optional{\metavar{MODE}} \\ --wrap\optional{=\metavar{MODE}}}]\itemend
  Blend around the boundaries of the panorama, or ``wrap
  around''.\genidx{wrap around}

  As this option significantly increases memory usage and computation
  time only use it, if the panorama will be

  \begin{itemize}
  \item
    consulted for any kind measurement, this is, all boundaries must match
    as accurately as possible, or

  \item
    printed out and the boundaries glued together, or

  \item
    fed into a virtual reality~(\acronym{VR})\genidx{virtual reality}%
    \gensee{VR@\acronym{VR}}{virtual reality} generator, which creates
    a seamless environment.
  \end{itemize}

  \noindent Otherwise, always avoid this option!

  With this option \App{} treats the set of input images (panorama) of
  width~$w$ and height~$h$ as an infinite data structure, where each
  pixel~$P(x, y)$ of the input images represents the set of
  pixels~$S_P(x, y)$.\footnotemark

  \footnotetext{Solid-state physicists will be reminded of the
    \ahref{\wikipediabornvonkarman}{\propername{Born}\genidx{Born@\propername{Born, Max}}-%
      \propername{von~K\'arm\'an}\genidx{Karman@\propername{von~K\'arm\'an, Theodore}}
      boundary condition}.}% FIXME

  \metavar{MODE} takes the following values:

  \begin{codelist}
  \item[\itempar{none \\ open}]\itemend This is a ``no-op''; it has
    the same effect as not giving \option{--wrap} at all.  The set of
    input images is considered open at its boundaries.

  \item[horizontal]\itemend
    Wrap around horizontally:
    \[
    S_P(x, y) = \{P(x + m w, y): m \in Z\}.
    \]

    This is useful for 360\angulardegree\genidx{panorama!360\angulardegree!horizontal}%
    \gensee{360@360\angulardegree{}!horizontal panorama}{panorama, 360\angulardegree}
    horizontal panoramas as it eliminates the left and right borders.

  \item[vertical]\itemend Wrap around vertically:
    \[
    S_P(x, y) = \{P(x, y + n h): n \in Z\}.
    \]

    This is useful for 360\angulardegree\genidx{panorama!360\angulardegree!vertical}%
    \gensee{360@360\angulardegree{}!vertical panorama}{panorama, 360\angulardegree}
    vertical panoramas as it eliminates the top and bottom borders.

  \item[\itempar{both \\ horizontal+vertical
      \\ vertical+horizontal}]\itemend Wrap around both horizontally
    and vertically:
    \[
    S_P(x, y) = \{P(x + m w, y + n h): m, n \in Z\}.
    \]

    In this mode, both left and right borders, as well as top and
    bottom borders, are eliminated.
  \end{codelist}

  Specifying \sample{--wrap} without \metavar{MODE} selects horizontal
  wrapping.
\end{codelist}

\genidx[\rangeendlocation]{advanced options}
