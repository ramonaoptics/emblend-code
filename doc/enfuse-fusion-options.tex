%% This file is part of Enblend.
%% Licence details can be found in the file COPYING.


\subsection[Fusion Options]{\label{sec:fusion-options}%
  \genidx[\rangebeginlocation]{fusion options}%
  \genidx{options!fusion}%
  Fusion Options}

Fusion options define the proportion to which each input image's pixel contributes to the output
image.

\begin{codelist}
  \label{opt:contrast-weight}%
  \optidx[\defininglocation]{--contrast-weight}%
  \genidx{weight!local contrast}%
  \gensee{contrast weight}{weight, local contrast}%
  \gensee{local contrast weight}{weight, local contrast}%
\item[--contrast-weight=\metavar{WEIGHT}]\itemend
  Sets the relative \metavar{WEIGHT} of high local-contrast pixels.

  Valid range: $\val{val:minimum-weight-contrast} \leq \metavar{WEIGHT} \leq
  \val{val:maximum-weight-contrast}$, default: \val{val:default-weight-contrast}.

  See \sectionName~\ref{sec:local-contrast-weighting} and \ref{sec:expert-options},
  option~\flexipageref{\option{--contrast-window-size}}{opt:contrast-window-size}.


  \label{opt:entropy-weight}%
  \optidx[\defininglocation]{--entropy-weight}%
  \genidx{weight!entropy}%
  \gensee{entropy weight}{weight, entropy}%
\item[--entropy-weight=\metavar{WEIGHT}]\itemend
  Sets the relative \metavar{WEIGHT} of high local entropy pixels.

  Valid range: $\val{val:minimum-weight-entropy} \leq \metavar{WEIGHT} \leq
  \val{val:maximum-weight-entropy}$, default: \val{val:default-weight-entropy}.

  See \sectionName~\ref{sec:expert-options} and \ref{sec:local-entropy-weighting},
  options~\flexipageref{\option{--entropy-window-size}}{opt:entropy-window-size} and
  \flexipageref{\option{--entropy-cutoff}}{opt:entropy-cutoff}.


  \label{opt:exposure-optimum}%
  \optidx[\defininglocation]{--exposure-optimum}%
  \genidx{exposure!optimum}%
  \gensee{optimum exposure}{exposure, optimum}%
\item[--exposure-optimum=\metavar{OPTIMUM}]\itemend
  Determine at what normalized exposure value the \metavar{OPTIMUM} exposure of the input images
  is.  This is, set the position of the maximum of the exposure weight curve.  Use this option
  to fine-tune exposure weighting.

  Valid range: $\val{val:minimum-exposure-optimum} \leq \metavar{OPTIMUM} \leq
  \val{val:maximum-exposure-optimum}$, default: \val{val:default-exposure-optimum}.


  \label{opt:exposure-weight}%
  \optidx[\defininglocation]{--exposure-weight}%
  \genidx{exposure!weight}%
  \gensee{weight!exposure}{exposure, weight}%
\item[--exposure-weight=\metavar{WEIGHT}]\itemend
  Set the relative \metavar{WEIGHT} of the ``well-exposedness'' criterion as defined by the
  chosen exposure weight function (see option~\flexipageref{\option{--ex\shyp
      po\shyp sure-weight-func\shyp tion}}{opt:exposure-weight-function}
  below).  Increasing this weight relative to the others will make well-exposed pixels
  contribute more to the final output.

  Valid range: $\val{val:minimum-weight-exposure} \leq
  \metavar{WEIGHT} \leq \val{val:maximum-weight-exposure}$, default:
  \val{val:default-weight-exposure}.

  See Section~\fullref{sec:exposure-weighting}.


  \label{opt:exposure-width}%
  \optidx[\defininglocation]{--exposure-width}%
  \genidx{exposure!width}%
  \gensee{width of exposure weight curve}{exposure, width}%
  \genidx{FWHM@\acronym{FWHM}}%
\item[--exposure-width=\metavar{WIDTH}]\itemend
  Set the characteristic \metavar{WIDTH} (\acronym{FWHM}) of the exposure weight function.  Low
  numbers give less weight to pixels that are far from the user-defined optimum
  (option~\flexipageref{\option{--exposure-optimum}}{opt:exposure-optimum}) and vice versa.  Use
  this option to fine-tune exposure weighting (See Section~\fullref{sec:exposure-weighting}).

  Valid range: $\metavar{WIDTH} > \val{val:minimum-exposure-width}$, default:
  \val{val:default-exposure-width}.


  \label{opt:hard-mask}%
  \optidx[\defininglocation]{--hard-mask}%
  \genidx{mask!hard}%
  \gensee{hard mask}{mask, hard}%
\item[--hard-mask]\itemend
  Force hard blend masks on the finest scale.  This is the opposite flag of
  option~\flexipageref{\option{--soft-mask}}{opt:soft-mask}.

  This blending mode avoids averaging of fine details (only) at the expense of increasing the
  noise.  However it considerably improves the sharpness of focus stacks.  Blending with hard
  masks has only proven useful with focus stacks.

  See also \sectionName~\ref{sec:fusion-options} and
  options~\flexipageref{\option{--contrast-weight}}{opt:contrast-weight} as well as
  \flexipageref{\option{--contrast-window-size}}{opt:contrast-window-size} above.


  \label{opt:saturation-weight}%
  \optidx[\defininglocation]{--saturation-weight}%
  \genidx{saturation weight}%
  \gensee{weight!saturation}{saturation weight}%
\item[--saturation-weight=\metavar{WEIGHT}]\itemend
  Set the relative \metavar{WEIGHT} of high-saturation pixels.  Increasing this weight makes
  pixels with high saturation contribute more to the final output.

  Valid range: $\val{val:minimum-weight-saturation} \leq \metavar{WEIGHT} \leq
  \val{val:maximum-weight-saturation}$, default: \val{val:default-weight-saturation}.

  Saturation weighting is only defined for color images; see
  \sectionName~\ref{sec:saturation-weighting}.


  \label{opt:soft-mask}%
  \optidx[\defininglocation]{--soft-mask}%
  \genidx{mask!soft}%
  \gensee{soft mask}{mask, soft}%
\item[--soft-mask]\itemend
  Consider all masks when fusing.  This is the default.
\end{codelist}

\genidx[\rangeendlocation]{fusion options}


%%% Local Variables:
%%% fill-column: 96
%%% End:
