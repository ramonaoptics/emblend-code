%% This file is part of Enblend.
%% Licence details can be found in the file COPYING.


\subsection[Common Options\commonpart]{\label{sec:common-options}%
  \genidx[\rangebeginlocation]{common options}%
  \genidx{options!common}%
  Common Options\commonpart}

Common options control some overall features of \App.  They are called ``common'' because they
are used most often.  However, in fact, \App{} and \OtherApp{} do have these options in common.

\begin{codelist}
  \label{opt:compression}%
  \optidx[\defininglocation]{--compression}%
  \genidx{output!file compression}%
  \genidx{compression}%
\item[--compression=\metavar{COMPRESSION}]\itemend
  Write a compressed output file.  The default is not to compress the output image.

  Depending on the output file format, \App{} accepts different values for
  \metavar{COMPRESSION}.

  \begin{description}
    \genidx{compression!\acronym{JPEG}}%
    \gensee{JPEG@\acronym{JPEG} compression}{compression}%
  \item[\acronym{JPEG} format.]\itemend
    The compression either is a literal integer or a keyword\hyp{}option combination.

    \begin{codelist}
      \gensee{JPEG@\acronym{JPEG} quality level}{compression}%
    \item[\metavar{LEVEL}]\itemend
      Set \acronym{JPEG} quality~\metavar{LEVEL}, where \metavar{LEVEL} is an integer that
      ranges from 0--100.

    \item[jpeg\optional{:\metavar{LEVEL}}]\itemend
      Same as above; without the optional argument just switch on standard \acronym{JPEG}
      compression.

      \genidx{compression!arithmetic \acronym{JPEG}}%
      \gensee{arithmetic \acronym{JPEG} compression}{compression}%
    \item[jpeg-arith\optional{:\metavar{LEVEL}}]\itemend
      Switch on arithmetic \acronym{JPEG} compression.  With optional argument set the
      arithmetic compression~\metavar{LEVEL}, where \metavar{LEVEL} is an integer that ranges
      from 0--100.
    \end{codelist}

  \item[\acronym{TIF} format.]\itemend
    Here, \metavar{COMPRESSION} is one of the keywords:

    \begin{codelist}
    \item[none]\itemend Do not compress.  This is the default.

      \genidx{compression!deflate}%
      \gensee{deflate compression}{compression}%
    \item[deflate]\itemend
      Use the \propername{Deflate} compression scheme also called
      \acronym{ZIP}-in-\acronym{TIFF}.  \propername{Deflate} is a lossless data compression
      algorithm that uses a combination of the \acronym{LZ77} algorithm and \propername{Huffman}
      coding.

      \genidx{compression!\acronym{JPEG} of \acronym{TIFF}}%
    \item[jpeg\optional{:\metavar{LEVEL}}]\itemend
      Use \acronym{JPEG} compression.  With optional argument set the
      compression~\metavar{LEVEL}, where \metavar{LEVEL} is an integer that ranges from 0--100.

      \genidx{compression!\acronym{LZW}}%
      \gensee{LZW@\acronym{LZW} compression}{compression}%
    \item[lzw]\itemend
      Use \propername{Lempel}-\propername{Ziv}-\propername{Welch} (\acronym{LZW}) adaptive
      compression scheme.  \acronym{LZW} compression is lossless.

      \genidx{compression!packbits}%
      \gensee{packbits compression}{compression}%
    \item[packbits]\itemend
      Use \propername{PackBits} compression scheme.  \propername{PackBits} is a particular
      variant of run-length compression; it is lossless.
    \end{codelist}

  \item[Any other format.]\itemend
    Other formats do not accept a \metavar{COMPRESSION} setting.  However, the underlying
    \uref{\hciiwrvigra}{\acronym{VIGRA}} library automatically compresses \filename{png}-files
    with the \propername{Deflate} method.  (\acronym{VIGRA} is the image manipulaton library
    upon which \App{} is based.)
  \end{description}


  \label{opt:gpu}%
  \optidx[\defininglocation]{--gpu}%
  \genidx{graphics processing unit}%
  \gensee{GPU@\acronym{GPU}}{graphics processing unit}%
  \genidx{central processing unit}%
  \gensee{CPU@\acronym{CPU}}{central processing unit}%
  \genidx{OpenCL@\acronym{OpenCL}}%
\item[--gpu \restrictednote{\acronym{OpenCL}-enabled versions
    only.}]\itemend
  Employ one of the graphics processing units (\acronym{GPU}s) to perform computing instead of
  the central processing units (\acronym{CPU}s) alone.  \App{} must have been compiled with
  support for \acronym{OpenCL} access to the \acronym{GPU} for this feature.

  Depending on the input images, the options passed to \App{}, and the relative performance of
  the \acronym{CPU}s to the \acronym{GPU}s this option may or may not increase performance.

  This option enables \acronym{GPU} processing on the selected \acronym{GPU}s.  To choose a
  particular \acronym{GPU} or override \App's default choice use
  \flexipageref{\sample{--prefer-gpu}}{opt:prefer-gpu}.

  Negate this option with \sample{--no-gpu}\optidx[\defininglocation]{--no-gpu}.


  \label{opt:levels}%
  \optidx[\defininglocation]{--levels}%
  \shoptidx{-l}{--levels}%
\item[\itempar{-l \metavar{LEVELS} \\ --levels=\metavar{LEVELS}}]\itemend
  Use at most this many \genidx{levels!pyramid}\gensee{pyramid levels}{levels,
    pyramid}\metavar{LEVELS} for pyramid\footnotemark{} blending if \metavar{LEVELS} is
  positive, or reduce the maximum number of levels used by $-\metavar{LEVELS}$ if
  \metavar{LEVELS} is negative; \sample{auto} or \sample{automatic} restore the default, which
  is to use the maximum possible number of levels for each overlapping region.

  \footnotetext{As \genidx{Jackson, Daniel@\propername{Jackson, Daniel}}\propername{Dr.~Daniel
      Jackson} correctly \ahref{\stargatewikiathetomb}{noted}, actually, it is not a pyramid:
    ``Ziggaurat, it's a \ahref{\wikipediaziggaurat}{Ziggaurat}.''}% FIXME

  The number of levels used in a pyramid controls the balance between local and global image
  features (contrast, saturation,~\dots) in the blended region.  Fewer levels emphasize local
  features and suppress global ones.  The more levels a pyramid has, the more global features
  will be taken into account.

  \begin{geeknote}
    As a guideline, remember that each new level works on a linear scale twice as large as the
    previous one.  So, the zeroth layer, the original image, obviously defines the image at
    single-pixel scale, the first level works at two-pixel scale, and generally, the $n$-th
    level contains image data at $2^n$-pixel scale.  This is the reason why an image of
    $\mbox{width} \times \mbox{height}$~pixels cannot be deconstructed into a pyramid of more
    than $\lfloor \log_2(\min(\mbox{width}, \mbox{height})) \rfloor$ levels.

    If too few levels are used, ``halos'' around regions of strong local feature variation can
    show up.  On the other hand, if too many levels are used, the image might contain too much
    global features.  Usually, the latter is not a problem, but is highly desired.  This is the
    reason, why the default is to use as many levels as is possible given the size of the
    overlap regions.  \App{} may still use a smaller number of levels if the geometry of the
    overlap region demands.
  \end{geeknote}

  Positive values of \metavar{LEVELS} limit the maximum number of pyramid levels.  Depending on
  the size and geometry of the overlap regions this may or may not influence any pyramid.
  Negative values of \metavar{LEVELS} reduce the number of pyramid levels below the maximum no
  matter what the actual maximum is and thus always influence all pyramids.  Use \sample{auto}
  or \sample{automatic} as \metavar{LEVELS} to restore the automatic calculation of the maximum
  number of levels.

  The valid range of the absolute value of \metavar{LEVELS} is \val{val:minimum-pyramid-levels}
  to \val{val:maximum-pyramid-levels}.

  \label{opt:output}%
  \optidx[\defininglocation]{--output}%
  \shoptidx{-o}{--output}%
  \genidx{filename!output}%
  \gensee{output filename}{filename, output}%
  \gensee{output filename!default}{filename, output, default}%
\item[\itempar{-o \metavar{FILE} \\ --output=\metavar{FILE}}]\itemend
  Place \appdoes{}ed output image in \metavar{FILE}.  If \sample{--output} is omitted, \App{}
  writes the resulting image to \genidx{filename!output!default}%
  \gensee{default output filename}{filename, output, default}%
  \filename{\val{val:default-output-filename}}.


  \label{opt:verbose}%
  \optidx[\defininglocation]{--verbose}%
  \shoptidx{-v}{--verbose}%
  \genidx{level!verbosity}%
  \gensee{verbosity level}{level, verbosity}%
\item[\itempar{-v \optional{\metavar{LEVEL}} \\ --verbose\optional{=\metavar{LEVEL}}}]\itemend
  Without an argument, increase the verbosity of progress reporting.  Giving more
  \option{--verbose}~options will make \App{} more verbose; see
  \sectionName~\fullref{sec:finding-out-details} for an examplary output.  Directly set a
  verbosity level with a non-negative integral~\metavar{LEVEL}.
  \tableName~\ref{tab:verbosity-levels} shows the messages available at a particular
  \metavar{LEVEL}.

  \begin{table}
    \centering
    \begin{tabular}{cp{.75\linewidth}}
      \hline
      \multicolumn{1}{c|}{Level} & \multicolumn{1}{c}{Messages} \\
      \hline\extraheadingsep
      0 & only warnings and errors \\
      1 & reading and writing of images \\
      2 & mask generation, pyramid, and blending \\
      3 & reading of response files, color conversions \\
      4 & image sizes, bounding boxes and intersection sizes \\
      5 & \restrictednote{\application{Enblend} only.} detailed
      information on the optimizer runs \\
      6 & estimations of required memory in selected processing steps \\
    \end{tabular}

    \caption[Verbosity levels]{\label{tab:verbosity-levels}Verbosity levels of \app{}; each
      level includes all messages of the lower levels.}
  \end{table}

  The default verbosity level of \App{} is
  \val{val:default-verbosity-level}.
\end{codelist}

\genidx[\rangeendlocation]{common options}


%%% Local Variables:
%%% fill-column: 96
%%% End:
