%% This file is part of Enblend.
%% Licence details can be found in the file COPYING.


\section[Option Delimiters\commonpart]{Option Delimiters\commonpart
  \label{sec:option-delimiters}
  \genidx{option delimiters}
  \genidx{options!delimiters}}

\application{Enblend} and \application{Enfuse} allow the arguments
supplied to the programs' options to be separated by different
separators.  The online documentation and this manual, however,
exclusively use the colon \sample{:} in every syntax definition and in
all examples.


\subsection[Numeric Arguments]{Numeric Arguments
  \label{sec:option-delimiters-numeric-arguments}
  \genidx{options!delimiters!numeric arguments}}

Valid delimiters are the the semicolon \sample{;}, the colon
\sample{:}, and the slash \sample{/}.  All delimiters may be mixed
within any option that takes numeric arguments.

Examples using some \application{Enfuse} options:

\begin{codelist}
\item[--contrast-edge-scale=0.667:6.67:3.5]\itemend Separate all
  arguments with colons.

\item[--contrast-edge-scale=0.667;6.67;3.5]\itemend Use semi-colons.

\item[--contrast-edge-scale=0.667;6.67/3.5]\itemend Mix semicolon and
  slash in weird ways.

\item[--entropy-cutoff=3\%/99\%]\itemend All delimiters also work in
  conjunction with percentages.

\item[--gray-projector=channel-mixer:3/6/1]\itemend Separate arguments
  with a colon and two slashes.

\item[--gray-projector=channel-mixer/30;60:10]\itemend Go wild and
  Enfuse will understand.
\end{codelist}


\subsection[Filename Arguments]{Filename Arguments
  \label{sec:option-delimiters-filename-arguments}
  \genidx{options!delimiters!filename arguments}}

Here, the accepted delimiters are \sample{,}, \sample{;}, and
\sample{:}.  Again, all delimiters may be mixed within any option that
has filename arguments.

Examples:

\begin{codelist}
\item[--save-masks=soft-mask-\%03i.tif:hard-mask-03\%i.tif]\itemend Separate
  all arguments with colons.

\item[--save-masks=\%d/soft-\%n.tif,\%d/hard-\%n.tif]\itemend Use a
  comma.
\end{codelist}
