%% This file is part of Enblend.
%% Licence details can be found in the file COPYING.


\chapter[Color Spaces\commonpart]{\label{sec:color-spaces}%
  \genidx[\rangebeginlocation]{colorspace}%
  \genidx{color profile}%
  Color Spaces And Color Profiles\commonpart}

\genidx{profile!\acronym{ICC}}%
\gensee{ICC@\acronym{ICC} profile}{profile, \acronym{ICC}}%
This chapter explains the connection of pixel data types,
\uref{\wikipediaiccprofile}{\acronym{ICC}}-color profiles, blend color spaces in \App{} or
\OtherApp.

\genidx{pixels!blending}%
\gensee{blending pixels}{pixels, blending}%
\genidx{pixels!fusing}%
\gensee{fusing pixels}{pixels, fusing}%
Here, we collectively speak of blending and do not distinguish fusing, for the basic operations
are the same.  Furthermore, we assume the multi-resolution spline algorithm has calculated a set
of weights~$w_i$ for $i = 1, 2, \dots$~and $\sum w_i = 1$ for each pixel that must be blended
from the participating input pixels~$P_i, i = 1, 2, \dots$.

\genidx{weighting factor}%
\genidx{luminance}%
In the simplest, non-trivial case we have to blend a pair of grayscale input pixels.  Given
their luminances~$L_1$, $L_2$ and their weighting factor~$0 \leq w \leq 1$, what luminance~$L$
is their ``weighted average''?  This is the heart of \App's and \OtherApp's pyramidal blending
operations!  We are in particular interested in a weighted average that appears \emph{visually}
correct, this is, our eyes and brains consider $L$ convincing or at the very least credible.

\begin{geeknote}
  \noindent Note that \App{} and \OtherApp{} face different obstacles in their respective
  domains of use.

  \begin{description}
  \item[\application{Enblend}]\itemend
    The overlapping areas usually are \emph{well matched} both geometrically and
    photometrically.  The differences of the pixels that must be blended are small.

  \item[\application{Enfuse} (using a Soft Mask\footnote{Fusing with a Hard Mask is different,
      because exactly one weight factor is unity and all the others are zero.  There is nothing
      to blend -- just to copy.})]\itemend
    The input images \emph{greatly differ} in exposure, saturation, or contrast.  This is
    exactly why we want to fuse them.  Thus, the luminance, saturation, and hue differences to
    be handled by \application{Enfuse} are generally quite high.
  \end{description}
\end{geeknote}

The details of blending pixels and in particular color pixels is quite intricate, which is why
we start this chapter with a mathematical introduction.


\section[Mathematical Preliminaries]{\label{sec:mathematical-preliminaries}%
  \genidx{color profiles!math}%
  Mathematical Preliminaries}

\genidx{luminance interval}%
\genidx{luminance interval!normalized}%
Let us first address grayscale images because they only require us to talk about luminances.
For a linear representation of luminances, we just blend for a given $t$ with
\begin{equation}\label{equ:trivial-luminance-blend}
  L = t L_1 + (1 - t) L_2 \quad \mbox{with} \quad 0 \leq t \leq 1,
\end{equation}
where the luminances~$L_i, i = 1, 2,$ range from zero to their data-type dependent maximum
value~$L_{\mathrm{max}}$, thereby defining a ``luminance interval''. We can always map this
interval to $(0, 1)$ by dividing the luminances by the maximum, which is why we call the latter
\emph{``normalized luminance interval''}:
\begin{equation}\label{equ:luminance-normalization}
  (0, L_{\mathrm{max}}) \rightarrow (0, 1)
\end{equation}
Obviously,
\begin{equation}\label{equ:normalized-luminance}
  0 \leq \bar{L} \leq 1
\end{equation}
holds for all values~$\bar{L} := L / L_{\mathrm{max}}$ in the normalized luminance interval.

\genidx{luminance interval!normalized}%
Sometimes images are gamma-encoded with exponent~$\gamma$ and the blended luminance becomes
\begin{equation}\label{equ:gamma-luminance-blend}
L' = \left(t L_1^{1/\gamma} + (1 - t) L_2^{1/\gamma}\right)^\gamma,
\end{equation}
\genidx{Brasseur@\propername{Eric Brasseur}}%
which couples $t$ and $L'$ in a non-linear way.  See also \propername{Eric Brasseur's}
explanation of the \uref{\ericbrasseurgamma}{gamma error in picture scaling}.

\begin{geeknote}
  \genidx{Lindbloom@\propername{Bruce Lindbloom}}%
  Typical gamma values are $\gamma = 2.2$ for \acronym{sRGB} and \acronym{AdobeRGB}, 1.8 for
  \acronym{AppleRGB}, and \acronym{ProPhotoRGB}, 1.0 for Linear Rec709 \acronym{RGB} and any
  others with ``linear'' in their names.  For an extensive overview check out \propername{Bruce
    Lindbloom's} \uref{\brucelindbloomworkingspaceinfo}{Information on Working Color Spaces.}
\end{geeknote}

The usual color-input images fed into \App{} are \acronym{RGB}-encoded, which means each pixel
comes as a triple of values~$\transpose{(r, g, b)}$ that represent the red, green, and~blue
parts.  We apply the normalization~\eqnref{equ:luminance-normalization} to each of the three
primary colors and arrive at an \emph{``\acronym{RGB}-cube''} with unit edge length.  The
vectors of primary colors span the cube
\[
  \vec{r} = \left(\begin{array}{c}1\\ 0\\ 0\end{array}\right),\quad
  \vec{g} = \left(\begin{array}{c}0\\ 1\\ 0\end{array}\right)\quad \mbox{and}\quad
  \vec{b} = \left(\begin{array}{c}0\\ 0\\ 1\end{array}\right).
\]
For each point inside -- familiarly called pixel -- the generalization of
\eqnref{equ:normalized-luminance} holds
\begin{equation}\label{equ:rgb-cube}
  \left(\begin{array}{c}0\\ 0\\ 0\end{array}\right) \leq
  \left(\begin{array}{c}r\\ g\\ b\end{array}\right) \leq
  \left(\begin{array}{c}1\\ 1\\ 1\end{array}\right).
\end{equation}

Blending the pixels of \emph{color} images is more complicated than blending plain luminances.
Although we can write down the na\"ive blending equation, \eqnref{equ:trivial-luminance-blend},
again for \acronym{RGB}-coded pixels
\[
  P_1 := \left(\begin{array}{c}r_1\\ g_1\\ b_1\end{array}\right)
  \quad \mbox{and}\quad
  P_2 := \left(\begin{array}{c}r_2\\ g_2\\ b_2\end{array}\right)
\]
and trivially arrive at
\begin{equation}\label{equ:trivial-rgb-blend}
  P :=
  \left(\begin{array}{c}r\\ g\\ b\end{array}\right) =
    t \left(\begin{array}{c}r_1\\ g_1\\ b_1\end{array}\right) +
    (1 - t) \left(\begin{array}{c}r_2\\ g_2\\ b_2\end{array}\right)
    \quad \mbox{with} \quad 0 \leq t \leq 1,
\end{equation}
but this means

\begin{itemize}
\item
  we implicitly treat the color components $r_i, g_i, b_i$ as \emph{separate} luminances, which
  they are not and moreover

\item
  we neglect the visual aspects, namely luminance, saturation, and~hue of the blended color
  pixel~$P$.
\end{itemize}


\section[Floating-Point Images]{\label{sec:floating-point-images}%
  \genidx{image!floating-point}%
  \gensee{floating-point image}{image, floating-point}%
  Floating-Point Images}

\genidx{transform!floating-point images}%
\genidx{image!\acronym{EXR}}%
\gensee{EXR@\acronym{EXR} image}{image, \acronym{EXR}}%
\genidx{image!floating-point \acronym{TIFF}}%
\gensee{floating-point \acronym{TIFF} image}{image, floating-point \acronym{TIFF}}%
\genidx{image!floating-point \acronym{VIFF}}%
\gensee{floating-point \acronym{VIFF} image}{image, floating-point \acronym{VIFF}}%
\gensee{VIFF@\acronym{VIFF} floating-point image}{image, floating-point \acronym{VIFF}}%
\genidx{transform!log}%
\gensee{log-transform}{transform, log}%
Floating-point images (\acronym{EXR}, floating-point \acronym{TIFF}, or \acronym{VIFF}) get a
special treatment.  Their values~$L$ are first converted by the \LogTransform-transform.

\begin{equation}\label{equ:log-transform}
  \mbox{Log}(L) :=
  \left\{
  \begin{array}{ll}
    1 + \log(1 + L) & \quad \mbox{for } L \geq 0 \mbox{ and} \\
    1 / (1 - L)     & \quad \mbox{otherwise,}
  \end{array}
  \right.
\end{equation}

\noindent which is undone by the inverse transform after blending.  Here, $\log(x)$ with a
lower-case initial denotes the natural logarithmic function (i.e.\ to base~$e$).
\figureName~\ref{fig:log-transform} shows the forward transform in the range from $-20$ to~100.
Around $L = 0$ function~$\mbox{Log}(L)$ has the series expansion
\[
\mbox{Log}(L) = 1 + L + \frac{L^2}{2} + O(L^3), \mbox{for } 0 \leq L < 1.
\]

\begin{figure}
  \centering
  \includeimage{log-transform}

  \caption[\LogTransform-transform]%
          {\label{fig:log-transform}\genidx{transform!log}%
            Forward \LogTransform-transform shown for the range of $-20 \leq L \leq 100$.  The
            domain of $\LogTransform(L)$ is the whole real axis.}
\end{figure}

This transform serves two purposes:

\begin{itemize}
\item
  During blending, even completely non-negative images can result in negative pixels.  A
  \LogTransform-transform followed by the inverse guarantees all-positive output.

\item
  For \acronym{HDR}~data, the \LogTransform-transform puts the samples closer to a perceptual
  space making the blending a little more pleasing.
\end{itemize}

In the current version of \App{} and \OtherApp{} it is \emph{strongly recommended} to use
blending inside the \acronym{RGB}-cube whenever the input data is in floating-point format; this
is the default, too.


\section[Color Profiles]{\label{sec:color-profiles}%
  \genidx{color profile}%
  Color Profiles}

\genidx{black-point}%
\gensee{ICC@\acronym{ICC} profile black-point}{black-point}%
\genidx{white-point}%
\gensee{ICC@\acronym{ICC} profile white-point}{white-point}%
\acronym{ICC}-color profiles completely absorb gamma
encodings~\eqnref{equ:gamma-luminance-blend} and \acronym{ICC}~profile aware software like
\App{} and \OtherApp{} decode and encode images automatically respecting the gamma curves.
Moreover color profiles define what is the darkest representable black, so called black-point
\[
  L = 0
  \quad \mbox{and}\quad
  \transpose{(r, g, b)} = \transpose{(0, 0, 0)}
\]
and analogously what is the purest and brightest white, the white-point
\[
  L = 1
  \quad \mbox{and}\quad
  \transpose{(r, g, b)} = \transpose{(1, 1, 1)}.
\]

\noindent By default, \App{} and \OtherApp{} expect that either

\begin{enumerate}
\item\label{enum:no-profile}
  no input image has a color profile or
\item\label{enum:same-profile}
  all images come with the \emph{same} \acronym{ICC}~profile.
\end{enumerate}

\genidx{image!black-and-white}%
\gensee{black-and-white image}{image, black-and-white}%
Even black-and-white images benefit from having attached appropriate profiles!

\genidx{color cube!\acronym{RGB}}%
\gensee{RGB@\acronym{RGB} color cube}{color cube, \acronym{RGB}}%
\genidx{color cube!\acronym{sRGB}}%
\gensee{sRGB@\acronym{sRGB} color cube}{color cube, \acronym{sRGB}}%
In Case~\ref{enum:no-profile}.\ the applications blend grayscale images in the normalized
luminance interval and color images inside the \acronym{sRGB}-cube.  To override the default
\acronym{sRGB}-profile select the desired profile with
option~\flexipageref{\option{--fallback-profile}}{opt:fallback-profile}.

In Case~\ref{enum:same-profile}.\ the images first are by default transformed to
\uref{\wikipediaciecam}{\acronym{CIELUV}} color space -- respecting the input color profile --
then they are blended or fused, and finally the data get transformed back to \acronym{RGB} color
space defined by the profile of the input images.  Consequently, the input profile is assigned
to the output image.  Enforce a different blending color space than \acronym{CIELUV} with
option~\flexipageref{\option{--blend-colorspace}}{opt:blend-colorspace}.

Mixing different \acronym{ICC}~profiles or alternating between images with profiles and without
them generates warnings as it generally leads to unpredictable results.

Floating-Point images are an exception to the above rules.  They are \emph{always} blended in
the \acronym{RGB} cube by default.  The next section describes their treatment in detail.


\section[Blending Color Spaces]{\label{sec:blending-color-spaces}%
  \genidx{colorspace!for blending and fusing}%
  \gensee{blending color space}{colorspace, for blending and fusing}%
  Blending Color Spaces}

\App{} and \OtherApp{} offer to work inside the \acronym{RGB}-cube~\eqnref{equ:rgb-cube} or in
several perceptually uniform color spaces.  To override the default select a particular blending
color space with option~\flexipageref{\option{--blend-colorspace}}{opt:blend-colorspace}.  Here
are the four available color spaces.

\begin{description}
  \genidx{color cube!\acronym{RGB}}%
  \gensee{RGB@\acronym{RGB} color cube}{color cube, \acronym{RGB}}%
\item[Identity Space / \acronym{RGB}-Color Cube]\itemend
  Calculate the blended pixel inside the luminance interval \eqnref{equ:trivial-luminance-blend}
  for grayscale images and inside the \acronym{RGB}-color cube as given in
  \eqnref{equ:trivial-rgb-blend}.

  This is the fastest color space to do computations within, i.e.\ it consumes by far the least
  computing power, because no transform to or from any of the perceptually uniform color spaces
  is done.

  \genidx{colorspace!\acronym{L*a*b*}}%
  \gensee{L*a*b*@\acronym{L*a*b*} colorspace}{colorspace, \acronym{L*a*b*}}%
  \genidx{white-point!\acronym{D50}}%
  \gensee{D50@\acronym{D50} white-point}{white-point, \acronym{D50}}%
\item[\urlmark{\wikipedialabcolorspace}{\acronym{L*a*b*}}]\itemend\urltext
  Represent each pixel as lightness~$L\raisedasterisk$, red-green difference~$a\raisedasterisk$,
  and yellow-blue difference~$b\raisedasterisk$.  The \acronym{L*a*b*} color space encompasses
  all perceivable colors.  It is completely independent of any device characteristics,
  approximates human vision, and is perceptually uniform.

  \App{} uses perceptual rendering intent and either the input profile's white-point or, if the
  \acronym{ICC}-profile lacks the \code{cms\-Sig\-Media\-White\-Point\-Tag}, fall back to the
  \acronym{D50} white-point (see,
  e.g.\ \uref{\wikipediastandardilluminant}{Standard illuminant}).

  The conversions from and to \acronym{L*a*b*} are moderately fast to compute; \acronym{L*a*b*}
  mode is two to three times slower than working within the \acronym{RGB}-color cube.

  \genidx{colorspace!\acronym{CIEL*u*v*}}%
  \gensee{CIEL*u*v*@\acronym{CIEL*u*v*} colorspace}{colorspace, \acronym{CIEL*u*v*}}%
\item[\urlmark{\wikipediacieluvcolorspace}{\acronym{CIEL*u*v*}}]\itemend\urltext
  Represent each pixel as lightness~$L\raisedasterisk$ and two color
  differences~$u\raisedasterisk$ and $v\raisedasterisk$.  Formulas of each are too complicated
  to show them here.

  The \acronym{L*u*v*} tries to be perceptually uniform in lightness as well as in color.

  The applications use the same rendering intent and white-point as with \acronym{L*a*b*}.

  The conversions from and to \acronym{L*u*v*} are almost as fast to compute as
  \acronym{L*a*b*}.

  \genidx{colorspace!\acronym{CIECAM02}}%
  \gensee{CIECAM02@\acronym{CIECAM02} colorspace}{colorspace, \acronym{CIECAM02}}%
  \genidx{rendering intent!perceptual}%
  \gensee{perceptual rendering intent}{rendering intent}%
  \genidx{white-point!\acronym{D50}}%
  \gensee{D50@\acronym{D50} white-point}{white-point, \acronym{D50}}%
\item[\urlmark{\wikipediaciecam}{\acronym{CIECAM02}}]\itemend\urltext
  Represent each pixel as lightness~$J$, chroma~$C$ (``colorfulness''), and hue
  angle~$h$.

  \begin{geeknote}
    Internally, the polar coordinates $(C, h)$ are translated to \propername{Cartesian}
    coordinates for the pyramids.
  \end{geeknote}

  The transformations to \acronym{CIECAM02} color space and back use perceptual rendering
  intent, the \acronym{D50} white point (see, e.g.\ \uref{\wikipediastandardilluminant}{Standard
    illuminant}), 500\dmn{lumen} surrounding light (``average'' in \acronym{CIECAM02} parlance),
  and assume complete adaption.

  Both \acronym{CIELUV} and \acronym{CIELAB} only model the color information generated for
  small and isolated color samples.  They cannot model the contextual effects of color
  perception.  However, \acronym{CIECAM02} can represent luminance adaptation, chromatic
  contrast and chromatic assimilation that arise in real world viewing conditions with
  heterogeneous, strongly contrasted, or three dimensional color sources.

  Computationally, \acronym{CIECAM02} is the most expensive blend color space.  If an
  appreciable number of pixels need additional refinement steps the speed of the transformation
  further drops.  Expect \acronym{CIECAM02} mode to be 8--800 times slower than blending within
  the \acronym{RGB}-color cube.
\end{description}

\noindent Surprisingly often blending ``inside the \acronym{RGB}-cube'' works, although
perceptually uniform color spaces, which represent luminance, saturation, and hue are preferable
for blending and fusing operations.


\section[Practical Considerations]{\label{sec:practical-considerations}%
  \genidx{colorspace!practical considerations}%
  Practical Considerations}

\begin{itemize}
\item
  For small projects stick with the default blend colorspaces.

\item
  For large projects switch on blending in the \acronym{RGB} color cube to speed up the assembly
  of the images.  When satisfied with all other parameters use one of the computationally more
  expensive, but perceptually uniform color spaces.

\item
  Banding is best fought by input images with a high bit depth (\mbox{$\geq 16$ bits} per
  channel).  A cheap and mostly vain trick is to force a large output bit depth with
  option~\flexipageref{\option{--depth}}{opt:depth}.  No blend color space can avoid banding if
  parts of the input images are almost ``monochrome''.

\ifenblend
  \item
    \restrictednote{\application{Enblend} only.} No color space can fix a seam-line gone
    haywire!  First re-run \application{Enblend} with
    \flexipageref{\sample{--visualize}}{opt:visualize} to inspect the seam-lines, then try one
    or more of the following

    \begin{itemize}
    \item
      Reorder the input images.

    \item
      Force blending with the full-resolution masks, using
      \flexipageref{\sample{--fine-mask}}{opt:fine-mask}.

    \item
      Generate the initial seam with the simpler and more robust nearest-feature transform
      (\acronym{NFT}) by employing option~\flexipageref{\option{--primary\hyp seam\hyp
          generator=nearest\hyp feature\hyp transform}}{opt:primary-seam-generator}.  See also
      \chapterName~\fullref{sec:seam-generators}.

    \item
      Disable the seam-line optimizer with \flexipageref{\sample{--no-optimize}}{opt:optimize}.
    \end{itemize}
\fi

\ifenfuse
  \item
    \restrictednote{\application{Enfuse} only.} No color space can fix blown highlights when
    fusing by exposure-weight; use
    \flexipageref{\sample{--exposure-cutoff}}{opt:exposure-cutoff} for this purpose and check
    whether saturation weight is zero.  \fi
\end{itemize}

\genidx[\rangeendlocation]{colorspace}


%%% Local Variables:
%%% fill-column: 96
%%% End:
